\documentclass[a4paper,12pt]{scrartcl} % From KOMA-script
\usepackage[margin=1in]{geometry}
\usepackage{xcolor}
\usepackage{graphicx}
\usepackage{fancyhdr}
\usepackage{lipsum} % For filler text
\usepackage{amsthm}
\usepackage{times}
\usepackage{microtype}
\usepackage{amsmath, amssymb, amsfonts, amssymb, float, enumitem}
\definecolor{darkbg}{HTML}{1E1E1E}
\definecolor{darktext}{HTML}{E0E0E0}
\definecolor{theoremcolor}{HTML}{d3dce5}
\definecolor{answercolor}{HTML}{e9dfc0}
\definecolor{sectioncolor}{HTML}{dec3c3}
\pagecolor{darkbg}
\color{darktext}
\newenvironment{solution}
  {\par\color{answercolor}\textbf{Solution:}\ }
  {\par}
\newenvironment{prf}
  {\par\textbf{Proof:}\ }
  {\par}

\newcounter{customcounter}
\newcommand{\setcustomcounter}[1]{\setcounter{customcounter}{#1}}

\newtheoremstyle{darktheorem}
  {\topsep}{\topsep}
  {\itshape\color{theoremcolor}}{}
  {\bfseries\color{theoremcolor}}{.}{.5em}{}
\theoremstyle{darktheorem}
\newtheorem{theorem}{Theorem}[section]
\newtheorem{lemma}[theorem]{Lemma}
\newtheorem{proposition}[theorem]{Proposition}
\newtheorem{corollary}[theorem]{Corollary}
\newtheorem{definition}[theorem]{Definition}
\newtheorem{example}[theorem]{Example}
\newtheorem{exercise}[customcounter]{Exercise}

\usepackage{fancyhdr}
\pagestyle{fancy}

% Clear default headers
\fancyhf{}
\renewcommand{\headrulewidth}{0pt}

% Set Author and Date in the top left
\fancyhead[L]{\textcolor{darktext}{\small Asher Christian \\ 006-150-286 \\ \today }}

\begin{document}


\title{\color{sectioncolor}HW 1 Math 151B}
\author{}
\date{}
\maketitle

% Apply fancyhdr on the title page
\thispagestyle{fancy}

\begin{exercise}
    Let $\lambda = a + ib$ where $a,b \in \mathbb{R}$ and $i = \sqrt{-1}$. The solution to
    IVP:
     \[
         y' = \lambda y \hspace{2cm} y(0) = 1
    .\] 
    is given by $y(t) = e^{\lambda t}$ show that the modulus of the solution is $|y(t)| = e^{at}$. Then, deduce
    that
    \[
    |y(t)|
    \begin{cases}
        \rightarrow 0 & a < 0\\
        = 1 & a = 0  \hspace{2cm} t \rightarrow \infty\\
        \rightarrow \infty & a > 0
    \end{cases}
    .\] 
    the modulus of a complex number $z = \alpha + i \beta$ is $|z| = \sqrt{\alpha^2 + \beta^2}$ and $e^{i\theta} = \cos(\theta) + i\sin(\theta)$
\end{exercise}
\begin{solution}
    \[
    y(t) = e^{\lambda t} = e^{(a + bi)t} = e^{at}(e^{bit}) = e^{at}(\cos(bt) + i\sin(bt))
    .\] 
    \[
    |y(t)| = e^{at}\sqrt{\cos^2(bt) + \sin^2(bt)} = e^{at}
    .\] 
    if $a < 0$ $at$ is monotone decreasing with no lower bound and $e^{-\infty}$ approaches zero so  
    \[
    \lim_{t\to \infty} e^{at} = 0
    .\] 
    if $a = 0$ then $e^{at} = e^{0} = 1$ for all $t$ and
    \[
    \lim_{t\to \infty}1 = 1
    .\] 
    if $a > 0$ then $at$ grows without bound and approaches $\infty$ with $e^{n}$ approaching infinity as well
    so 
    \[
    \lim_{t\to \infty}e^{at} = \infty
    .\] 
\end{solution}

\begin{exercise}
    \leavevmode
    \begin{enumerate}[label = (\alph*)]
        \item Let $D = \{(t,y) : a \le t \le b,c \le y \le d\}$. Suppose $f \in C^{1}(D)$.
            Show that $f$ is Lipschitz continuous in $D$.
        \item Show that $f(t,y) = \frac{t^2y^2}{1+t^2}$ is Lipschitz continuous in $D = \{(t,y): t_0 \le t \le t_f, -\delta \le y \le \delta\}$
    \end{enumerate}
\end{exercise}
\begin{solution}
    \begin{enumerate}[label = (\alph*)]
        \item  $f \in C^{1}(D)$ implies that
            \[
            \frac{df}{dt} = f_t
        .\] exists and is continuous on $D$.
        since $D$ is a compact set, and $f_t$ is continuous
        \[
            \sup_{p \in D}|f_t(p)| = M
        .\] 
        exists.\\
        consider the following for any $y_1 < y_2 \in [c,d]$
            \[
                \frac{f(t,y_1) - f(t,y_2)}{y_1-y_2} = f_t(t,\xi)
            .\] 
            for some $\xi \in (y_1,y_2)$ and in particular
            \[
            |f(t,y_1) - f(t,y_2)| = |y_1-y_2||f(t,\xi)| \le M|y_1-y_2|
            .\] 
            and so $f$ is Lipschitz continuous on $D$
        \item 
            \[
            |\frac{df}{dt}| = 2y^2|\frac{t}{(1+t^2)^2}| \le 2y^2 \le 2\delta^2
            .\] 
            this is because $|\frac{t}{(1+t^2)^2}| \le 1$ which can be verified by considering the cases where $|t| < 1$
            and $|t| \ge 1$ and since  $|y| \le \delta$ by nature of $D$, and so by the nature
            of the previous part of the question, $f$ is lipschitz continuous.
    \end{enumerate}
\end{solution}


\end{document}
