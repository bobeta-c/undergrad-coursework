\documentclass{amsart}
\usepackage{amsfonts} % For math fonts
\usepackage{amsmath, amssymb, amsthm}
\usepackage{float}
\usepackage{enumitem}
\usepackage{graphicx}
\setlist[enumerate,1]{label=\arabic*.}
\setlist[enumerate,2]{label=\alph*.,itemindent=2em}
\setlist{topsep=0pt, leftmargin=*, labelsep=1em}


\usepackage{listings}
\usepackage{xcolor}

\lstset{
    language=Python,
    backgroundcolor=\color{black}, % Light gray background
    basicstyle=\ttfamily\small\color{white}, % Code style
    keywordstyle=\color{cyan}\bfseries, % Keywords style
    stringstyle=\color{yellow}, % Strings style
    commentstyle=\color{gray}, % Comments style
    frame=single, % Box around code
    rulecolor=\color{white}, % Frame color
    numbers=left, % Line numbers
    numberstyle=\tiny\color{white}, % Line number style
    breaklines=true, % Automatic line breaking
    showstringspaces=false
}


\title{HW 6 - 151A}
\author{Asher Christian 006-150-286}
\date{ 26.02.25} 

\begin{document}
    \maketitle
    \section{Exercise 1}
    \emph{
        The Trapezoidal Rule applied to $\int_{0}^{2}f(x)dx$ gives the value 4
        and Simpson's rule gives the value 2. What is $f(1)$ t
    }\\
    Trapezoidal approximation is $f(0) + f(2) = 4 \rightarrow \frac{1}{3}f(0) + \frac{1}{3}f(2) = \frac{4}{3}$ 
    Simpsons approximation is $\frac{1}{3}f(0) + \frac{4}{3}f(1) + \frac{1}{3}f(2) = 2$ thus
    \[
    \frac{4}{3}f(1) = 2 - \frac{4}{3} = \frac{2}{3} \implies f(1) = \frac{1}{2}
    .\] 

    \section{Exercise 2}
    \emph{
        Let $b-a = 2h$, $x_1 = a_h = \frac{a+b}{2}$ using Taylor's polynomial of second order
        \[
        f(x) = f(x_1) + (x-x_1)f'(x_1) + \frac{1}{2}(x-x_1)^2f''(\eta (x))
        .\] 
        to derive the error for the midpoint rule
        \[
        \int_{a}^{b}f(x)dx = 2hf(x_1) + \frac{h^{3}}{3}f''(\xi)
        .\] 
        where $\xi \in [a,b]$
    }
    Take the integral of both sides
    \[
    \int_{a}^{b}f(x) = \int_{a}^{b}f(x_1) + f'(x_1)\int_{a}^{b}(x-x_1) + \frac{1}{2}\int_{a}^{b}(x-x_1)^2f''(\eta (x))
    .\] 
    appllying the mean value theorem for integrals and since
    \[
        \int_{a}^{b}(x-\frac{a+b}{2}) = \int_{-h}^{h}udu= (\frac{u^2}{2})|^{h}_{-h} = h^2-\frac{-h^2}{2} = 0
    .\] 
    we get
    \[
    \int_{a}^{b}f(x) = f(x_1)2h + \frac{f''(\xi)}{2}\int_{-h}^{h}u^2du
    .\] 
    and
    \[
        \int_{-h}^{h}u^2 = \frac{u^{3}}{3}\Big|^{h}_{-h}  = \frac{2h^{3}}{3}
    .\] 
    \[
    =2hf(x_1) + \frac{f''(\xi)}{2}\frac{2h^{3}}{3} = 2hf(x_1) + \frac{h^{3}}{3}f''(\xi)
    .\] 

    \section{Exercise 3}
    \emph{
        Use the Composite Trapezoidal and Composite Simpson's rules to approximate the integra
        \[
        \int_{1}^{2}x\ln(x)dx
        .\] 
        with $n=3$ subintervals. What are the relative errors?
    }
    The true integral is
    \[
    \int_{1}^{2}x\ln(x) = \frac{2^2}{2}\ln(2) - \frac{1}{2}\int_{1}^{2}xdx = 2\ln(2) - \frac{3}{4} \approx 0.63629436112
    .\] 
    for trapezoidal rule
    \[
        \int_{a}^{b}f(x)dx \approx \frac{1}{6}(f(1) + 2f(1 +\frac{1}{3}) + 2f(1 + \frac{2}{3}) + f(2)) = 0.642699772257
    .\] 
    with relative error
    \[
        \frac{|0.642699772257 - \int_{1}^{2}x\ln(x)dx|}{\int_{1}^{2}x\ln(x)dx} =  0.0100667419499
    .\] 
    for composite simpson's rule we must have an additional interval so everything is $n=4$
    \[
    \int_{1}^{2}f(x)dx \approx \frac{1}{9}(f(1) + 2f(1 + \frac{1}{2}) + 4(f(1 + \frac{1}{4})+f(1+\frac{3}{4}) + f(2)) = 0.636309829797
    .\] 
    with relative error
    \[
    0.0000243105675671
    .\] 

    \section{Exercise 4}
    \emph{
        Recall that the error in the Composite Trapezoidal rule (CTR) using n subintervals of width
        h is given by
        \begin{equation} \label{eq:1}
            -\frac{h^2}{12}(b-a)f''(\mu)
        \end{equation}
        for some $\mu \in (a,b)$
        \begin{enumerate}[label = (\alph*)]
            \item \label{seq:1}
                Determine the values of $n$ and $h$ that are sufficient to approximate
                \begin{equation}
                    \int_{1}^{2}x\ln(x)dx
                \end{equation}
                to within an error tolerance of $\tau = 10^{-5};$ that is, determine $n$ so that
                the error when applying the CTR to $(2)$ is smaller in absolute value than $\tau$ 
            \item Repeat part \ref{seq:1} for the case of Composite Simpson's Rule.
        \end{enumerate}
    }
    Solution:
    \begin{enumerate}[label = (\alph*)]
        \item first note that 
            \[
            f''(\mu) = \frac{1}{\mu}
            .\] 
            which is strictly positive and bounded from above on the interval $\mu \in (1,2)$ by $f''(1) = 1$
            thus the error in absolute value is less than
             \[
            \frac{h^2}{12}(2-1)1 = \frac{h^2}{12}
            .\] 
            thus we must find $h$ such that $\frac{h^2}{12} < 10^{-5}$, the largest $h$ that ensures this bound
            is $h = 0.0109544511501$. This h or smaller can be obtained by dividing into $n = 92$ or more subintervals
        \item The simpson's rule error is given by
            \[
            \frac{h^{4}}{180}(b-a)f^{(4)}(\xi)
            .\] 
            $f^{(4)}(\mu) = \frac{2}{\mu^{3}}$ strictly positive and bounded above by $2$
            thus the error is given by
             \[
            \frac{h^{4}}{90}
            .\] 
            the largest $h$ value that works is $h = 0.173205080757$ corresponding to $n \ge 6$
    \end{enumerate}

    \section{Exercise 5}
    \emph{
        find the constants a,b,c, and d such that the quadrature rule below has a degree of exactness
        at least 3:
        \[
        \int_{-1}^{1}f(x)dx = af(-1) +bf(1) + cf'(-1) + df'(1)
        .\] 
    }
    if $f(x) = x_0 + x_1x + x_2x^2 + x_3x^{3}$ then
    \[
        \int_{-1}^{1}f(x)dx = 2x_0+\frac{2}{3}x_2 = a(x_0-x_1+x_2-x_3) + b(x_0+x_1+x_2+x_3) + c(x_1-2x_2+3x_3) + d(x_1+2x_2+3x_3)
    .\] 
    giving us four equations
    \[
    a + b = 2
    .\] 
    \[
    -a+b+c+d = 0
    .\] 
    \[
    a+b-2c+2d = \frac{2}{3}
    .\] 
    \[
    -a+b+3c+3d = 0
    .\] 
    this results in 
    \[
    a = b = 1, c = \frac{1}{3}, d = -\frac{1}{3}
    .\] 

    \section{Exercise 6}
    \emph{
        Consider the nonlinear equation for x:
        \[
        \int_{0}^{x}\frac{1}{\sqrt{2\pi}}e^{-\frac{t^2}{2}}dt = 0.45
        .\] 
        \begin{enumerate}[label = (\alph*)]
            \item Define 
                \[
                f(x) := \int_{0}^{x}\frac{1}{\sqrt{2\pi }}e^{-\frac{t^2}{2}}dt-0.45
                .\] 
                Using the Fundamental Theorem of Calculus write down Newton's method applied to
                the equation $f(x) = 0$
            \item Each step of Newton's method derived in (a) requires an evaluation of f(x),
                while we cannot evaluate f(x). Instead, we apply Composite Trapezoidal Rule to estimate f(x)
                with N subintervals. Write down the expression for Composite Trapezoidal Rule with N subintervals
                to approximate $\int_{0}^{x}\frac{1}{\sqrt{2\pi }}e^{-\frac{t^2}{2}}$ (denoted by $I_{\text{trap}}[g;x;N]$ where $g(t)$ is the function
                we are integrating. Then write down a modified version of Newton's method where the integration is approximated by its composite Trapezoidal Rule.
            \item Write a code to imploement the method you derived above with $N = 50$ subintervals and 
                initialization $x_0 = 0.5$. Stop the method you derived above when the approximation for the residual $|f(x_n)|$ is 
                smaller than $\tau = 10^{-5}$ or a maximum of iteration 100 is reached. Report the number of iteration and the 
                approximated root (with 5 sig figs)
        \end{enumerate}
    }
    \begin{enumerate}[label = (\alph*)]
        \item 
            \[
                 x_n = x_{n-1} - \frac{f(x_{n-1})}{f'(x_{n-1})}
            .\] 
            but 
            \[
            \frac{d}{dx}\int_{0}^{x}\frac{1}{\sqrt{2\pi }}e^{-\frac{t^2}{2}}dt =\frac{1}{\sqrt{2\pi} }e^{-\frac{x^2}{2}}
            .\] 
            so
            \[
                x_n = x_{n-1} - \frac{\int_{0}^{x_{n-1}}\frac{1}{\sqrt{2\pi} }e^{-\frac{t^2}{2}}dt - 0.45}{\frac{1}{\sqrt{2\pi }}e^{-\frac{x_{n-1}^2}{2}}}
            .\] 
        \item $I_{\text{trap}}[g;x;N]$ is
            \[
            \frac{h}{2}(g(0) + 2\sum_{i=1}^{N-1}g(x_i) + g(x))
            .\] 
            where $h = \frac{x}{N}$. applying $g(0) = \frac{1}{\sqrt{2\pi}}$ we get
            \[
                I_{\text{trap}}[g;x;N] = \frac{x}{2N}(\frac{1}{\sqrt{2\pi}}+2\sum_{i=1}^{N-1}g(i\frac{x}{N}) + g(x))
            .\] 
            plugging this in to the newtons method we get
            \[
                x_n = x_{n-1} - \frac{\frac{x_{n-1}}{2N}(\frac{1}{\sqrt{2\pi}} + 2\sum_{i=1}^{N-1}g(ix_{n-1}/N) + g(x_{n-1}))}{\frac{1}{\sqrt{2\pi}}e^{\frac{-x_{n-1}^2}{2}}}
            .\] 

        
    \end{enumerate}


\end{document}
