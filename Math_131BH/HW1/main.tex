\documentclass{amsart}
\usepackage{amsfonts} % For math fonts
\usepackage{amsmath, amssymb, amsthm}
\usepackage{float}
\usepackage{enumitem}
\usepackage{xcolor}
\usepackage{graphicx}
\setlist[enumerate,1]{label=\arabic*.}
\setlist[enumerate,2]{label=\alph*.,itemindent=2em}
\setlist{topsep=0pt, leftmargin=*, labelsep=1em}

\title{HW 1 - 131BH}
\author{Asher Christian 006-150-286}
\date{ 21.01.25}

\begin{document}
    \maketitle
    \section{Exercise 1.2}
    \emph{If $f: \mathbb{R}^2 \setminus \{0,0)\} \rightarrow \mathbb{R}$, three limits we can consider are
        \[
        \lim_{y\to 0}\lim_{x\to 0}f(x,y), \;\;\; \lim_{x\to 0}\lim_{y\to 0}f(x,y), \;\;\; \lim_{(x,y)\to (0,0)}f(x,y)
        .\] 
        Compute these limits if they exist for
        \[
        f(x,y) = \frac{xy}{x^2+y^2}, \;\;\; f(x,y) = \frac{x^2-y^2}{x^2+y^2}
        .\] 
    }

    \section{Exercise 1.3}
    \emph{Find a sequence of functions $f_n: [0,1] \rightarrow \mathbb{R}$ that converges to the
        zero function and such that the sequence $( \int_{0}^{1}f_n(x)dx)_n$, increases without bound.
    }\\
    Let $(f_n)_n$ be defined such that
    \[
        f_n : [0,1] \rightarrow \mathbb{R} :=
    \begin{cases}
        -e^{n}(x)(x-\frac{1}{n}) & 0 \le x \le \frac{1}{n}\\
        0 & \frac{1}{n} < x \le 1
    \end{cases}
    .\] 
     \[
    \lim_{n\to \infty}f_n(x) = 0
    .\] 
    because for each $x \in [0,1]$ if $x = 0$ it is always zero, and if $x \ne 0$ then pick $n > \frac{1}{x}$ which makes $f_n(x) = 0$ 
    Thus the function converges to zero for each point in $[0,1]$.\\
    The integral:
    \[
    \int_{0}^{1}f_n(x)dx = \int_{0}^{\frac{1}{n}}-e^{n}(x^2-\frac{x}{n}) = -e^{n}\int_{0}^{\frac{1}{n}}x^2-\frac{x}{n} 
    .\] 
    since $x^2-\frac{x}{n}$ is continuous, by the fundamental theorem of calculus we have
    \[
    \int_{0}^{\frac{1}{n}}x^2-\frac{x}{n} = \frac{(\frac{1}{n})^{3}}{3}-\frac{(\frac{1}{n})^2}{2n} - 0 + 0 = n^{-3}(\frac{1}{3}-\frac{1}{2}) = -\frac{1}{6}n^{-3}
    .\] 
    and
    \[
        \int_{0}^{1}f_n(x)dx = \frac{1}{6}e^{n}n^{-3} = \frac{e^{n}}{6n^{3}}
    .\] 
    \[
        \frac{e^{1}}{1^{3}} > 1
    .\] 
    \[
    \frac{e^{n+1}}{(n+1)^{3}} = \frac{e^{n}}{n^{3}} \frac{e}{\frac{(n+1)^{3}}{n^{3}}} = \frac{e^{n}}{n^{3}} (\frac{n}{n+1})^{3}e  > \frac{e^{n}}{n^{3}} (\frac{4}{5})^{3}e = \frac{e^{n}}{n^{3}} \frac{64}{125}e > \frac{e^{n}}{n^{3}} \frac{128}{125}
    .\] 
    for $n > 4$ since $\frac{n}{n+1} = 1 $

    and so the integral is unbounded and its limit does not exist.

    \section{Exercise 1.5}
    \emph{Show that if $\sum_{n=1}^{\infty}a_n$ is a convergent series of real numbers and $(v_k)_k$ 
        is a subsequence of $(n)_{n=1}^{\infty}$ then
        \[
            (a_1+...+a_{v_1}) + (a_{v_1+1} + ... + a_{v_2}) + (a_{v_2+1} + ... + a_{v_3}) + ... = \sum_{n=1}^{\infty}a_n
        .\] 
    }\\
    Let $(s_n)_n$ be a sequence with each $s_n$ defined by $(s_n) = (a_{v_{n-1}} + a_{v_{n-1}+1} + ... + a_{v_n})$ taking  $v_0$ to be $1$ 
    We aim to show that $ \sum_{n=1}^{\infty}a_n = \sum_{n=1}^{\infty}s_n$\\
    First we show that $\sum_{n=1}^{\infty}s_n$ exists. For this let $N \in \mathbb{N}$ be such that the partial sums $\sum_{i=1}^{n}a_i, \sum_{i=1}^{m}a_i$ are less than $\epsilon$
    apart when $n,m > N$
    Pick $N'$ s.t.  $v_{N'} > N$ then any two partial sums 
    \[
        \sum_{i=1}^{n}s_i - \sum_{i=1}^{m}s_i = s_{n+1} + s_{n+2} + ... + s_m = a_{v_n} + ... + a_{v_m}
    .\] 
    assuming $n < m$ and this is the same form as a cauchy sequence for the original sum and thus is less than epsilon and the sum exists.
    \\
    To show that the two sums are equal, for any $\epsilon > 0$ pick $N_1$ such that the partial sums of $N_1$ or more terms  of $a_i$ are within $\frac{\epsilon}{3}$ of eachother and pick $N_2$ such that $s_i$ sums are the same and set $N = max\{N_1,N_2\}$
    \[
        \sum_{i=1}^{\infty}a_i = \sum_{i=1}^{N}a_i + \sum_{i=N}^{\infty}a_i
    .\] 
    \[
        \sum_{i=1}^{\infty}s_n = \sum_{i=1}^{N}s_i + \sum_{i=N}^{\infty}s_i
    .\] 
    \[
    |\sum_{i=1}^{\infty}a_i - \sum_{i=1}^{\infty}s_i| \le |\sum_{i=1}^{N}s_i - \sum_{i=1}^{N}a_i| + |\sum_{i=N}^{\infty}a_i| + |\sum_{i=N}^{\infty}s_i| \le \frac{\epsilon}{3} + \frac{\epsilon}{3} + \frac{\epsilon}{3} = \epsilon
    .\] 
    Since $(v_n)_n$ is a subsequence of  $ \mathbb{N}$ and by the definition of $s_n$, $\sum_{i=1}^{N}s_i$ is equivalent to $\sum_{i=1}^{v_{N}}a_i$ and $v_{N} \ge N$ so the first term is
    equivalent to the cauchy statement made previously and thus less than $\frac{\epsilon}{3}$.
    Consider additionally the partial sums of the last two terms. Each partial sum is equivalent to  $\sum_{i=1}^{m}x_i - \sum_{i=1}^{N}x_i$ with $x_i \in \{a_i, s_i\}$ and thus is also a cauchy difference since  $m,N \ge N$ and so those terms
    too are less than $\frac{\epsilon}{3}$ justifying the last step. Therefore the two sums are within $\epsilon$ of eachother for any $\epsilon$ and are therefore the same.

    \section{Exercise 1.6}
    \emph{Let $(a_n)_n \subset [0,+\infty)$ be a sequence of positive numbers which is monotone
        non increasing. Show that the following hold.
        \\
        (i) If $\sum_{n=1}^{\infty}a_n$ is convergent then $\lim_{n\to +\infty}na_n =0$. \\
        (ii) $\sum_{n=1}^{\infty}a_n$ is convergent if and only if $\sum_{n=1}^{\infty}2^{n}a_{2^{n}}$ is convergent.\\
    }
    Since the series is convergent it is also cauchy in particular for any $\epsilon > 0$ there exists $N_\epsilon \in \mathbb{N}$ s.t.
    \[
    |\sum_{i=1}^{n}a_i - \sum_{i=1}^{m}a_i| < \epsilon
    .\] 
    whenever $n,m \ge N_\epsilon$
    To show the limit converges to zero, for any epsilon pick $N = 2(N_\epsilon+1)$ for $\frac{\epsilon}{2}$ as before and for any $n > N$ consider $m =  \frac{1}{2}n$ flooring m if odd and $n-1$
    \[
        \frac{\epsilon}{2} \ge |a_{\frac{n}{2}} + a_{\frac{n}{2}+1} + ... + a_{n}| = \sum_{i=1}^{n}a_i - \sum_{i=1}^{\frac{n}{2}-1}a_i \ge |a_{n} + a_{n} + ... + a_{n}| \ge \frac{1}{2}n|a_{n}| 
    .\] 
    The last part is an inequality because of the case where $\frac{n}{2}$ is floored and an extra term is included thus overcounting by $\frac{1}{2}a_n$
    multiplying through by $2$ on both sides we get
    \[
    \epsilon \ge n|a_n|
    .\] 
    for any $n > N$ and thus the limit of $n|a_n|$ is equal to zero proving $(i)$ \\
    For $(ii)$ first note that
     \[
         \sum_{n=1}^{\infty}2^{n}a_{2^{n}} \le a_1 + \sum_{n=1}^{\infty}2^{n}a_{2^{n}} = \sum_{n=0}^{\infty}2^{n}a_{2^{n}} 
    .\] which also converges since the two differ by a constant $a_1$
    Expanding this out we see every partial sum
    \[
        \sum_{i=0}^{n}2^{i}a_{2^{i}} = a_1 + a_2 + a_2 + a_4 + a_4 + a_4 + a_4 + ... + a_{2^{n}} \ge a_1 + a_2 + a_3 + a_4 + a_5 + a_6 + a_7 + ... + a_{2^{n+1}-1} = \sum_{i=0}^{2^{n+1}-1}a_i
    .\] 
    This is due to the fact that $a_n$ is monotonic non-increasing. Noting that the every term
    of the orignal series is positive so the partials are increasing and thus we have shown that they are bounded from below by 
    the partials of the $2^{n}$ series which converges thus this series converges proving the first direction that the convergence of $2^{n}a_{2^{n}}$ implies the convergence
    of $a_n$. to show the second direction that $a_n$ convergence implies $2^{n}a_{2^{n}}$ convergence.\\
    To show this first we note that since $\sum_{i=1}^{\infty}a_i$ converges absolutely, for all $n > N$ $\frac{|a_{n+1}|}{|a_n|} < \xi$ with $\xi \in [0,1)$
    This implies then for all $n$ s.t. $2^{n} > N$
    \[
        \frac{|a_{2^{n+1}}|}{|a_{2^{n}}|} < \xi^{2n}
    .\] 
    Since there are $2^{n}$ elements between the two values and so
    \[
        \frac{|2^{n+1}a_{2^{n+1}}|}{|2^{n}a_{2^{n}}|} < 2 \xi^{2n}
    .\] 
    Increasing  $N$ so $\xi^{2N} < \frac{1}{2}$ (which is posisble because the value converges to zero as n goes to infinity) we see that for this new (possibly larger) $N$ 
    the series defined by $\sum_{i=1}^{\infty}2^{i}a_{2^{i}}$ satisfies the ratio test and therefore converges proving the second direction.

    \section{Exercise 1.7}
    \emph{Integral Test) Let $f: [1, +\infty) \rightarrow \mathbb{R}$ be a monotone non increasing
        function. Prove that the following are equivalent.
        \begin{enumerate}
            \item (i) $\sum_{n=1}^{\infty}f(n)$ is convergent
            \item (ii) $\lim_{n\to +\infty} \int_{1}^{n}f$ exists
        \end{enumerate}
    }
    First to prove $(i)$ implies $(ii)$ consider the two functions $f_1,f_2 : [1,N] \rightarrow \mathbb{R}$
    \[
    f_1(x) = f(\lfloor x \rfloor), f_2(x) = f(\lceil x \rceil)
    .\] 
    For arbitrary $N$
    And note that since $N$ is finite, $f_1$ and $f_2$ are step functions. additionall
    \[
        \forall x \in [0,N] \;\;\; f_2(x) \le f(x) \le f_1(x)
    .\] 
    By the monotonicity of $f$.
    The integrals of $f_1$ and $f_2$ are well defined and in particular
    \[
    \int_{1}^{N}f_1(x)dx = \sum_{n=1}^{N-1}f(n)
    .\] 
    \[
    \int_{1}^{N}f_2(x)dx = \sum_{2}^{N}f(n)
    .\] 
    And
    \[
    \int_{1}^{N}f_2(x)dx \le \int_{1}^{N}f(x)dx \le \int_{1}^{N}f_1(x)dx
    .\] 
    But the integrals of $f_1$ and $f_2$ are bounded since the sums are bounded and since
    we can assume each $f(x) \ge 0$ since by the sum converging $f(x)$ must approach $0$. 
    So the integrals are bounded above by the infinite sum $\sum_{n=1}^{\infty}f(n)$, below by 0 and are monotone increasing.
    Additionally since $f$ is montone and bounded it is integrable. so the integral exists for each $N$ and since the partial sums below and 
    above are cauchy, the integral itself must be cauchy and so it converges and the limit exists.
    Seen below
    for any $\epsilon > 0$ pick $N$ such that the lower and upper sums are cauchy within $\epsilon$. for all $m,n > N$
    Then
     \[
         \int_{m}^{n}f_2(x)dx \le \int_{m}^{n}f(x)dx \le \int_{m}^{n}f_1(x)dx
    .\] 
    \[
    -\epsilon \le \int_{m}^{n}f(x)dx \le \epsilon
    .\] 
    Additionally since the limit is over all real numbers note that since the integral is strictly increasing a bound that works for all integer differences greater than $N$ will work
    for all real numbers greater than $N$\\
    Now to prove $(ii)$ implies $(i)$. Assume that $\lim_{n\to \infty}\int_{1}^{n}f$ exists. Let $I_n = \int_{n-1}^{n}f$ Then $\int_{1}^{n}f = \sum_{2}^{n}I_n$.
    Since $f$ is monotone nonincreasing it achieves its maximum on the interval $[n-1,n]$ at $n-1$ and its minimum at $n$. In particular $|I_n| \ge |f(n)|$ since $f(n)$ is nonnegative.
    Thus by the comparison test if  $\lim_{n\to \infty}\int_{1}^{n}f = \lim_{n\to \infty}\sum_{2}^{n}I_n$  exists, by the comparison test $\sum_{1}^{\infty}f(n)$ exists and converges absolutely
    \section{Exercise 1.8}
    \emph{
        For which $p > 0$ the following series converge:
        \[
        \sum_{n=1}^{\infty}\frac{1}{n^{p}},  \;\; \sum_{n=2}^{\infty}\frac{1}{n(logn)^{p}}, \sum_{n=3}^{\infty}\frac{1}{nlog(n)(loglog(n))^{p}}
        .\] 
    }
    All elements of these series are strictly posiive for large $N$ so if they converge the converge absolutely. So it suffices to check if they converge absolutely.\\
    Additionally these sums are all montoone non-increasing once the denominators are positive allowing many of the following techniques.
    For any $p > 1$
    \[
    \sum_{n=1}^{\infty}\frac{1}{n^{p}}
    .\]
    converges by the integral test for if $p \ne 1$ then
    \[
    \lim_{n\to \infty}\int_{1}^{n}\frac{1}{x^{p}}dx = \lim_{n\to \infty}\frac{n^{1-p}}{1-p} - \frac{1}{1-p}
    .\] 
    The non-constant term of this limit converges to 0 if $p > 1$ and diverges towards infinity if  $p < 1$. In the case $p = 1$ 
    we have shown in class that this is the harmonic series and it diverges.
    Using the result from Exercise 1.6 we see that the second sum converges if and only if
    \[
        \sum_{n=1}^{\infty}2^{n}a_{2^{n}} = \sum_{n=1}^{\infty}\frac{2^{n}}{2^{n}(log(2^{n}))^{p}} = \sum_{n=1}^{\infty} \frac{1}{n^{p}log(2)^{p}} = \frac{1}{log(2)^{p}} \sum_{n=1}^{\infty} \frac{1}{n^{p}}
    .\] 
    converges We have shown this in the previous question to only converge when $p > 1$
    Similarly for the last series using the same rule
    \[
    \sum_{n=1}^{\infty}\frac{2^{n}}{2^{n}log(2^{n})(log(log(2^{n})))^{p}} = \frac{1}{log(2)}\sum_{n=1}^{\infty} \frac{1}{n(log(nlog(2)))^{p}}
    .\] 
    This series also only converges if the following converges
    \[
    \sum_{n=1}^{\infty} \frac{2^{n}}{2^{n}(log(2^{n}log(2)))^{p}} = \sum_{n=1}^{\infty}\frac{1}{(nlog(2) + loglog(2))^{p}} = \frac{1}{(log(2))^{p}}\sum_{n=1}^{\infty}\frac{1}{(n+\frac{loglog(2)}{log(2)})^{p}}
    .\] 
    This series only converges when $p > 1$ since by the integral test and u substitution as before this is simply a shifted version of the first example.
    \section{Exercise 1.9}
    \emph{
        On the set $ \mathbb{R} \setminus \{-1,-2,...\}$ show the convergence of the series
        \[
        \sum_{n=1}^{\infty}(\frac{1}{n}-\frac{1}{n+x}) = \sum_{n=1}^{\infty}\frac{n+x-n}{n(n+x)} = x\sum_{n=1}^{\infty}\frac{1}{n^2+nx}
        .\] 
    }
        If $x \ge 0$ then by the comparison test this series converges since $\sum_{n=1}^{\infty}\frac{1}{n^2}$ converges.
        If $x < 0$  for any $\epsilon > 0$ pick $N$ such that if $n_2 > n_2 > N$ the series $\frac{1}{n^2}$ is cauchy within $\epsilon$. i.e.
        \[
        \sum_{n=n_1}^{n_2}\frac{1}{n^2} < \epsilon
        .\] 
        then pick $M = N + |x|$
        The for the same  $n_2,n_1$
        \[
        \sum_{n=n_1+|x|}^{n_2+|x|}\frac{1}{n^2+nx} = \frac{1}{(n_1+|x|)^2-(n_1+|x|)|x|} + ... + \frac{1}{(n_2+|x|)^2-(n+|x|)|x|}
        .\] 
        and for any $n > 0$ 
        \[
        \frac{1}{(n+|x|)^2-(n+|x|)|x|} = \frac{1}{n^2+2n|x|+|x|^2-n|x|-|x|^2} = \frac{1}{n^2+n|x|} < \frac{1}{n^2}
        .\] 
        so within this new $M$, the series is cauchy. Thus for any $x$ and any $\epsilon > 0$ there exists some $M$ such that for any
        $n,m > M$ the partial sums up to $n$ and $m$ are within $\epsilon$ apart and the series converges. The series does not converge uniformly however
        since picking  $\epsilon = \frac{1}{2}$ for any $N$ such that if $n_1,n_2 > N$
        \[
        \sum_{n=n_1}^{n_2}\frac{1}{n} - \frac{1}{n+x} < \epsilon
        .\] 
        pick $x = -N+\frac{1}{2}$ Then the term
        \[
        \frac{1}{N} - \frac{1}{N-N + \frac{1}{2}} = \frac{1}{N} + 2
        .\] 
        Is clearly greater than $\frac{1}{2}$ and so the series does not converge uniformly.

        \section{Exercise 1.10}
        \emph{Root test: let $\sum_{n=1}^{\infty}a_n$ be a series of real numbers such that there exists
            $r \in (0,1)$ such that $\sqrt[n]{|a_n|} \le r$ for all sufficiently large $n$. Show that $\sum_{n=1}^{\infty}a_n$ is
            absolutely convergent.
        } \\
        \[
        |a_n|^{\frac{1}{n}} \le r \rightarrow |a_n| \le r^{n}
        .\] 
        chopping off the first terms until the inequality holds
        \[
        \sum_{n=n_1}^{\infty}|a_n| \le \sum_{n=n_1}^{\infty}r^{n}
        .\] 
        In particular the second series converges since it is the geometric series with each partial sum equal to
        \[
            \frac{1-r^{n}}{1-r} 
        .\] 
        The limit of which is well defined. And so $a_n$ in series converges absolutely by comparison test.
        \section{Exercise 1.11}
        \emph{
            Prove that if $\sum_{n=1}^{\infty}a_n$ and $\sum_{n=1}^{\infty}b_n$ are absolutely convergent
            series of real numbers then the series $\sum_{m,n=1}^{\infty}a_nb_m$ is also absolutely convergent
            and 
            \[
            \sum_{m,n=1}^{\infty}a_nb_m = (\sum_{n=1}^{\infty}a_n)(\sum_{n=1}^{\infty}b_n)
            .\] 
        }\\
        Let $s_i = \sum_{m,n=1}^{i}a_nb_m = a_1b_1+a_1b_2+...+a_1b_i+a_2b_1+...+a_ib_i$
        $\lim_{i\to \infty}s_i$ exists if and only if the partial sums converge
        that is
        \[
        |\sum_{m,n=n_1}^{n_2}a_nb_m| < \epsilon
        .\] 
        For $n_1,n_2 > N$
        For any $\epsilon > 0$ pick $N_1$ such that $\sum_{n=n_1}^{n_2}|a_n| < \sqrt{\epsilon}$ when $n_1,n_2 > N_1$ and likewise $N_2$ such that
        the same holds for the sequence $b_n$. Let $N = \max\{N_1,N_2\}$ Then if $n_2 > n_1 \ge N$
        \[
            \sum_{m,n = n_1}^{n_2}|a_nb_m| = |a_{n_1}b_{n_1}| + |a_{n_1}b_{n_1+1}| + ... + |a_{n_2}b_{n_2}|
        .\] 
        \[
            = |a_{n_1}|\sum_{n_1}^{n_2}|b_m| + |a_{n_1+1}|\sum_{n_1}^{n_2}|b_m| +... + |a_{n_2}|\sum_{n_1}^{n_2}|b_m|
        .\] 
        The $b_n$ sum is constant in this expression so it simplifies further to
        \[
        = (\sum_{m=n_1}^{n_2}|b_n|)(\sum_{n=n_1}^{n_2}|a_n|) \le \sqrt{\epsilon} \sqrt{\epsilon} = \epsilon
        .\] 
        And so the series converges absolutely.
        Additionally every partial sum as can be seen previously is exactly equal to the above expression
        regardless of the lower and upper bounds picked. Thus every partial sum is equal to the equation and so the limits
        since
        \[
        s_i = A_iB_i 
        .\] 
        and
        \[
        \lim_{i\to \infty}A_i,  \;\;\; \lim_{i\to \infty}B_i
        .\] 
        exist it implies that
        \[
        \lim_{i\to \infty}s_i = \lim_{i\to \infty}A_i \lim_{i\to \infty}B_i
        .\] 
        \section{Exercise 1.12}
        \emph{
            Let $(c_n)_{n=0}^{\infty} \subset \mathbb{R}$. Prove that the radius of convergence of the power series 
            $\sum_{n=1}^{\infty}c_nx^{n}$ is $\frac{1}{\limsup_{n\to \infty}|c_n|^{\frac{1}{n}}}$
        }\\
        if $\limsup_{n\to \infty}c_n = \infty$ then clearly $R = 0$ because we can always find a $c_n$ arbitrarily large so for this proof assume $R \ne 0$
        First note that a power series converges absolutely within its radius of convergence.
        Now to show that if $R = \frac{1}{\limsup_{n\to \infty}|c_n|^{\frac{1}{n}}}$  for any $-R < x < R$ considering
        the absolute sum. And noting that $|c_n|$ is bounded let $M = sup_{n}|c_n|$
         \[
        \sum_{n=1}^{\infty}|c_nx^{n}| \le \sum_{n=1}^{\infty}|c_n|(\frac{1}{\limsup_{n\to \infty}|c_n|^{\frac{1}{n}} + \epsilon})^{n} \le \sum_{n=1}^{\infty}(\frac{|c_n|^{\frac{1}{n}}}{|c_n|^{\frac{1}{n}}+\epsilon})^{n} 
        \] 
        \[
        = \sum_{n=1}^{\infty}(\frac{1}{1+\frac{\epsilon}{|c_n|^{\frac{1}{n}}}})^{n} \le \sum_{n=1}^{\infty}(\frac{1}{1+\frac{\epsilon}{M^{\frac{1}{n}}}})^{n} \le \sum_{n=1}^{\infty}(\frac{1}{1+\frac{\epsilon}{M}})^{n}
        .\] 
        Assuming $M \ge 1$ otherwise, replace  $\frac{\epsilon}{M}$ with $\epsilon$ in the last statement.
        Either way this is clearly a power series and so converges absolutely. Thus proving that the radius of convergence holds.
        Now to prove that $R$ is a the maximum such bound for the radius of convergence.
        Assume for contradiction that there exists some $R' > R$ that can serve as a radius of convergence.
        Pick $R < x < R' $ and consider the subsequence  $(c_{n_k})_k$ such that $\limsup_{n\to \infty}|c_n|^{\frac{1}{n}} = \lim_{k\to \infty}|c_{n_k}|^{\frac{1}{n_k}}$
        \[
            \sum_{n=1}^{\infty}c_{n_k}x^{n_k} \le \sum_{n=1}^{\infty}c_nx^{n}
        .\] 
        \[
            \sum_{k=1}^{\infty}|c_{n_k}x^{n_k}| = \sum_{k=1}^{\infty}|c_{n_k}||\frac{1}{\lim_{k\to \infty}|c_{n_k}^{\frac{1}{n_k}}|-\epsilon}|^{n_k}
        .\] 
        Consider the individual terms and $N > 0$ such that $|c_{n_k}^{\frac{1}{n_k}} - \lim_{k\to \infty}c_{n_k}^{\frac{1}{n_k}}| < \frac{\epsilon}{2}$
         \[
             |\frac{c_{n_k}^{\frac{1}{n_k}}}{\lim_{k\to \infty}|c_{n_k}|^{\frac{1}{n_k}}-\epsilon}|^{n_k} \ge |\frac{c_{n_k}^{\frac{1}{n_k}}}{c_{n_k}^{\frac{1}{n_k}}-\frac{\epsilon}{2}}|^{n_k} > 1
         .\] 
         Thus every element after $N$ terms is greater than 1 so the partial sums of this new series do not converge.
         Additionally there are infinitely many elements corresponding to these $c_{n_k}$ in the original series
         so for arbitrary partial sums greater than $N$ there are some that have an element of  $c_{n_k}$ in them and thus are in absolute value
         larger than 1. Therefore the original series does not converge this is a contradiction which shows that $R$ is the least upper bound on the radius of convergence. 
\end{document}
