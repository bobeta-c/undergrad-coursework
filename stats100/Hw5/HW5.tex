\documentclass{article}
\usepackage{amsmath} % For math equations
\usepackage{amsfonts} % For math fonts
\usepackage{amssymb} % For math symbols
\usepackage{enumitem}
\setlist[enumerate,1]{label=\arabic*.}
\setlist[enumerate,2]{label=\alph*.,itemindent=2em}

\title{Homework 5}
\author{Asher Christian 006-150-286}
\date{2024-05-25}

\begin{document}
    \maketitle
    \section{Problem 1}
    Suppose $X \sim f(x)$, and let $Y = aX + b$, where $a$ and $b$ are constants.
    \begin{enumerate}
        \item
            Prove $E(Y) = aE(X) + b$, and $Var(Y) = a^2Var(X)$.
            \begin{align*}
                E(Y) &= \int_{-\infty}^{\infty} Y(x)f(x)dx\\ 
                     &= \int_{-\infty}^{\infty}(aX+b)f(x)dx \\ 
                     &= a\int_{-\infty}^{\infty}xf(x)dx + b\int_{-\infty}^{\infty}f(x)dx \\
                     &= aE(x) + b
            \end{align*}
            \begin{align*}
                Var(Y) &= \int_{-\infty}^{\infty}(Y(x)-E(Y))^2f(x)dx \\
                       &= \int_{-\infty}^{\infty}(aX+b-aE(x)-b)^2f(x)dx \\
                       &= a^2\int_{-\infty}^{\infty}(X-E(X))^2f(x)dx \\
                       &= a^2Var(X)
            \end{align*}
        \item
            Assuming $a > 0$, calculate the density of $Y , g(y)$
            $X = \frac{Y-b}{a}$ and $\frac{dX}{dY} = \frac{1}{a}$ 
            \[
            g(y) = f(x)*\frac{dx}{dy} = \frac{f(\frac{y-b}{a})}{a}
            .\] 
    \end{enumerate}
    \section{Problem 2}
    Suppose $Z \sim N(0,1)$, i.e.,
    \[
        f(z) = \frac{1}{\sqrt{2 \pi }} e^{-\frac{z^2}{2}}
    .\] 
    \begin{enumerate}
        \item Calculate, $E(Z), Var(Z), E(|Z|)$
             \begin{align*}
                 E(Z) &= \int_{-\infty}^{\infty}\frac{x}{\sqrt{2\pi}}e^{-\frac{x^2}{2}}dx\\
                      &= -\frac{1}{\sqrt{2\pi}}\int_{-\infty}^{\infty}e^{u}du \\
                      &= -\frac{1}{\sqrt{2\pi}}  \left. e^{-\frac{x^2}{2}} \right|_{-\infty}^{\infty}\\
                      &= -\frac{1}{\sqrt{2\pi}} + \frac{1}{\sqrt{2\pi}} \\
                      &= 0
            \end{align*}
            \begin{align*}
                Var(Z) &= E(Z^2) - E(Z)^2 \\
                       &= E(Z^2) \\
                E(Z^2) &= \int_{-\infty}^{\infty}\frac{z^2}{2\pi}e^{-\frac{z^2}{2}}dx \\
                       &= \frac{2}{\sqrt{\pi}}\int_{-\infty}^{\infty}x^2e^{-x^2}dx \\
                       &= \frac{4}{\sqrt{\pi}}\int_{0}^{\infty}x^2e^{-x^2}dx\\
                       &= \frac{2}{\sqrt{\pi}} \int_{0}^{\infty}z^{\frac{3}{2}-1}e^{-z}dz \\
                       &= \frac{2}{\sqrt{\pi}}\frac{\sqrt{\pi}}{2} \\
                       &= 1
            \end{align*}
            \begin{align*}
                E(|Z|) &= \int_{-\infty}^{\infty}\frac{|x|}{\sqrt{2\pi}}e^{-\frac{x^2}{2}}dx \\
                       &= \frac{2}{\sqrt{2\pi}}\int_{0}^{\infty}xe^{-\frac{x^2}{2}}dx \\
                       &= -\frac{\sqrt{2}}{\sqrt{\pi}} \left. e^{-\frac{x^2}{2}} \right|_0^{\infty}\\
                       &= 0 + \frac{\sqrt{2}}{\sqrt{\pi}} \\
                       &= \frac{\sqrt{2}}{\sqrt{\pi}}
            \end{align*}
        \item Let $X = \mu + \sigma Z$. Calculate the density of  $X g(x), E(X), Var(X)$ based on Problem 1.
            \begin{align*}
                g(x) &= \frac{\frac{1}{\sqrt{2\pi}}e^{-\frac{(\frac{x-\mu}{\sigma})^2}{2}}}{\sigma}\\
                     &= \frac{1}{\sigma\sqrt{2\pi}}e^{-\frac{1}{2}(\frac{x-\mu}{\sigma})^2}
            \end{align*}
            \begin{align*}
                E(X) &= \sigma E(Z) + \mu \\
                     &= \mu
            \end{align*}
            \begin{align*}
                Var(X) &= \sigma^2Var(Z) \\
                       &= \sigma^2
            \end{align*}
        \item Suppose $P(Z \in [-2,2]) = 95\%$ then what is $P(X \in [\mu - 2\sigma, \mu + 2\sigma])$?
            \begin{align*}
                95\% &= \int_{-2}^{2}\frac{1}{\sqrt{2\pi}}e^{-\frac{z^2}{2}}dx \\
                z &= \frac{x-\mu}{\sigma}\\
                dz &= \frac{1}{\sigma}\\
                x_1 &= \mu + 2\sigma\\
                x_0 &= \mu - 2\sigma\\
                95\% &= \int_{\mu - 2\sigma}^{\mu + 2\sigma}\frac{1}{\sigma\sqrt{2\pi}}e^{-\frac{1}{2}(\frac{x-\mu}{\sigma})^2}dx
            \end{align*}
    \end{enumerate}
    \section{Problem 3}
    Poisson process: Suppose we divide the time axis into small periods (0, $\Delta $t),
($\Delta $t, 2$\Delta $t), .., (t, t + $\Delta $t), ... Suppose within each interval, we flip a coin independently. Suppose
the probability of getting a head is $\lambda$$\Delta $t. Let T be the time until the first head. Let X be the
number of heads within [0, t]
    \begin{enumerate}
        \item
            Find the probability density function of $T$ , and calculate $E(T)$ based on Geometric distribution.
            \begin{align*}
                P(T \in (t, t+\Delta t)) &= (1-\lambda \Delta t)^{\frac{t}{\Delta t }}\lambda \Delta t\\
                                         &= (e^{-\lambda \Delta t })^{\frac{t}{\Delta t }}\lambda \Delta t\\
                                         &= e^{-\lambda t}\lambda \Delta t \\
                                         &\sim e^{-\lambda t}\lambda  dt 
            \end{align*}
            \begin{align*}
                E(T) &= \int_{0}^{\infty}\lambda te^{-\lambda t}dt\\
                     &= \frac{1}{\lambda }\int_{0}^{\infty}ue^{-u}du\\
                   w &= u \\
                  dw &= 1 \\
                  dv &= e^{-u}\\
                   v &= -e^{-u} \\
                   \int_{0}^{\infty}ue^{-u}du &= \left. -ue^{-u} \right|_{0}^{\infty} + \int_{0}^{\infty}e^{-u}du\\
                                              &= 0 + \left. -e^{-u} \right|_{0}^{\infty} \\
                                              &= 1 \\
                       E(T) &= 1 * \frac{1}{\lambda }\\
                            &= \frac{1}{\lambda }
            \end{align*}
        \item Calculate $E(X)$ based on Binomial distribution. Find the probability mass function $P (X = k)$ as $\Delta  t \rightarrow 0$.
            \[
            t = n\Delta t 
            .\] 
            \[
            n = \frac{t}{\Delta t }
            .\] 
            \[
            E(X) = np = \frac{t}{\Delta t }\lambda \Delta t = t\lambda  
            .\] 
            \begin{align*}
                P(X \in (x, x + \Delta x)) &= \binom{n}{x}(\lambda \Delta t)^{x}(1-\lambda \Delta t)^{n-x} \\
                                           &= \frac{n!}{(n-x)!x!}(\lambda \Delta t)^{x}(1-\lambda \Delta t)^{n-x} \\
                                           &= \frac{\frac{t}{\Delta t }!}{(\frac{t}{\Delta t }-x)!x!}(\lambda \Delta t)^{x}(1-\lambda \Delta t)^{\frac{t}{\Delta t }-x} \\
                                           &= \frac{(\frac{t}{\Delta t })(\frac{t}{\Delta t }-1)...(\frac{t}{\Delta t } - x +1)}{x!}(\lambda \Delta t)^{x}(1-\lambda \Delta t)^{\frac{t}{\Delta t }}(1-\lambda \Delta t)^{-x}\\
                                           &= \frac{(t)(t-\Delta t)(t-2\Delta t)...(t-(x-1)\Delta t)}{x!}(\lambda )^{x}(1-\lambda \Delta t)^{\frac{t}{\Delta t }}(1-\lambda \Delta t)^{-x}\\
                                           &\rightarrow \frac{(\lambda t)^{x}}{x!}e^{-\lambda t}
            \end{align*}
            the approximation is as $\Delta t $ approaches 0 making many terms vanish
    \end{enumerate}
    \section{Problem 4}
    Brownian motion or diffusion: Suppose a particle starts from 0, and within each
    period, it moves forward or backward by $\Delta x$, each with probability 1/2. Let $X_t$ be the position at
    time $t$ (assuming $t$ is a multiple of $\Delta  t$). Suppose there are $n$ periods within $[0, t]$, i.e., $\Delta  t = t/n$.
    Then we can write
    \[
    X_t=\sum_{i=1}^{n}\epsilon_i\Delta x 
    .\] 
    where $P(\epsilon_i = 1) = P(\epsilon_i = -1) = \frac{1}{2}$ and $Z_i$ are independent
    \begin{enumerate}
        \item
            Calculate $E(X_t)$ and $Var(X_t)$.\\
            Let $Y \sim $ Binomial($n, \frac{1}{2}$)
            \[
            X_t = (Y - (n - Y))\Delta x  = (2Y - n)\Delta x  = (2Y - \frac{t}{\Delta t})\Delta x
            .\] 
            \begin{align*}
                E(X_t) &= (2E(Y) - \frac{t}{\Delta t })\Delta x \\
                       &= (\frac{2t}{2\Delta t } - \frac{t}{\Delta t })\Delta x \\
                       &= 0\\
                Var(X_t) &= 4(\Delta X)^2Var(Y) \\
                         &= 4(\Delta x)^2 n(\frac{1}{4}) \\
                         &= n(\Delta x)^2 \\
                         &= \frac{t(\Delta x)^2}{\Delta t }\\
            \end{align*}
        \item What is the relationship between $\Delta  x$ and $\Delta  t$ so that $Var(X_t)$ does not depend on discretization?
            \[
                \frac{(\Delta x)^2}{\Delta t } = \sigma^2
            .\]
            $\sigma^2$ is a constant
            \[
            \Delta x = \sigma \sqrt{\Delta t }
            .\] 
        \item According to the central limit theorem, what is the distribution of $X_t$?
           According to the central limit theorem, as t increases, the distribution approaches a normal distribution centered around 0 with Variance $\sigma^2t$ 
    \end{enumerate}
    \section{Problem 5}
    \begin{enumerate}
        \item Suppose we flip a fair coin 100 times independently. Let $X$ be the number of heads. Based on normal approximation, find the 95\% probability interval for $X$.
            \[
                X \sim \text{Binomial}(n, \frac{1}{2}), n = 100
            .\] 
            \begin{align*}
                \mu &= \frac{n}{2} \\
                \sigma^2 &= \frac{n}{4}\\
                \sigma &= \frac{\sqrt{n}}{2}\\
            \end{align*}
            Let $Z = \frac{X-\mu }{\sigma } = \frac{X-\frac{n}{2}}{\frac{\sqrt{n}}{2}}$\\
            and $P(Z \in (a,b)) \rightarrow \int_{a}^{b}f(z)dz$ with $f(z) = \frac{1}{\sqrt{2\pi }}e^{-\frac{z^2}{2}}$
            \begin{align*}
                95\% &= \int_{-a}^{a} f(z)dz\\
                     &= \left. \frac{\text{erf}(\frac{x}{\sqrt{2}})}{2} \right|_{-a}^{a} \\
                1.9 &= \text{erf}(\frac{a}{\sqrt{2}}) - \text{erf}(-\frac{a}{\sqrt{2}})\\
                a &\approx 2\\
                \pm2 &= \frac{x-\frac{n}{2}}{\frac{\sqrt{n}}{2}}\\
                x &= \pm\sqrt{n}+\frac{n}{2} = 60,40
            \end{align*}
            To Verify
            \[
            P(X \in (40, 60)) = \sum_{i=40}^{60}\frac{\binom{100}{i}}{2^{100}} \approx 0.964799799782
            .\] 
        \item Suppose 20\% of the population support a candidate $A$. Suppose we randomly sample 100 people for the population (with replacement). Let $\hat{p} = X/100$ be the proportion of people in the sample who support candidate $A$. Based on normal approximation, find the 95\% probability interval for $\hat{p}$.
            \[
                X \sim \text{Bimomial}(100, \frac{1}{5}), \mu = 20, \sigma^2 = 16
            .\] 
            \[
                \hat{p} = \frac{X}{100}, \mu = 0.2, \sigma^2 = 0.0016, \sigma = 0.04
            .\] 
            \[
                Z = \frac{\hat{p} - 0.2}{0.04} \rightarrow P(Z \in (-2, 2)) \approx 95\%
            .\] 
            \[
                P(\hat{p} \in (0.12, 0.28)) = 95\%
            .\] 
        \item Suppose we randomly throw 10,000 points into the unit square $[0, 1]^2$. Let $A$ be the region $x^2 + y^2 \le 1$. Let m be the number of points that fall into $A$. Let $\hat{\pi } = 4m/10000$ be our Monte Carlo estimate of $\pi $. What is the approximate normal distribution of $\hat{\pi }$? What is the 95\% probability interval of $\hat{\pi }$?
            \[
                m \sim \text{Bimomial}(10,000,\frac{\pi}{4}), \mu = \frac{10,000\pi }{4}, \sigma = 100\sqrt{\frac{\pi }{4}(1-\frac{\pi }{4})}
            .\] 
            \[
                \hat{\pi } = \frac{4m}{10,000}, \mu = \pi , \sigma = \frac{\sqrt{\pi (1-\frac{\pi }{4})}}{50}
            .\] 
            \[
                \pm 2=\frac{x-\pi }{\frac{\sqrt{\pi (1-\frac{\pi }{4})}}{50}}
            .\] 
            \[
                x = \pm \frac{1}{25}\sqrt{\pi (1-\frac{\pi }{4})} + \pi 
            .\] 
            \[
                P(\hat{\pi } \in (-\frac{1}{25}\sqrt{\pi (1-\frac{\pi }{4})} + \pi, \frac{1}{25}\sqrt{\pi (1-\frac{\pi }{4})} + \pi)) \approx 95\%
            .\] 
    \end{enumerate}
    \section{Problem 6}
    \begin{enumerate}
        \item Negative binomial distribution\\
            \[
            P(X=k) = \binom{k+r-1}{k}(1-p)^{k}p^{r}
            .\] 
        \item Hyper-Geometric
            \[
            P(X=k) = \frac{\binom{K}{k}\binom{N-K}{n-k}}{\binom{N}{n}}
            .\] 
        \item zipf
            \[
            P(x) = \frac{x^{-(\rho+1)}}{\zeta (\rho +1)}
            .\] 
        \item Chi-square
            \[
            f(x;k) =
            \begin{cases}
            \frac{x^{\frac{k}{2}-1}e^{-\frac{x}{2}}}{2^{\frac{k}{2}}\Gamma (\frac{k}{2})}, & x > 0;\\
                0, & \text{otherwise.}
            \end{cases}
            .\] 
        \item student t
            \[
            {\displaystyle \textstyle \ {\frac {\Gamma \left({\frac {\ \nu +1\ }{2}}\right)}{{\sqrt {\pi \ \nu \ }}\ \Gamma \left({\frac {\nu }{\ 2\ }}\right)}}\ \left(\ 1+{\frac {~x^{2}\ }{\nu }}\ \right)^{-{\frac {\ \nu +1\ }{2}}}\ }
            .\] 
        \item Cauchy
            \[
            {\displaystyle f(x;x_{0},\gamma )={1 \over \pi }\left[{\gamma  \over (x-x_{0})^{2}+\gamma ^{2}}\right],}
            .\] 
        \item Gamma
            \[
            {\displaystyle f(x)={\frac {\beta ^{\alpha }}{\Gamma (\alpha )}}x^{\alpha -1}e^{-\beta x}}
            .\] 
        \item Beta
            \[
            {\displaystyle {\frac {x^{\alpha -1}(1-x)^{\beta -1}}{\mathrm {B} (\alpha ,\beta )}}\!}
            .\] 
            \[
            {\displaystyle \mathrm {B} (\alpha ,\beta )={\frac {\Gamma (\alpha )\Gamma (\beta )}{\Gamma (\alpha +\beta )}}}
            .\] 
        \item Weibull
            \[
            {\displaystyle f(x;\lambda ,k)={\begin{cases}{\frac {k}{\lambda }}\left({\frac {x}{\lambda }}\right)^{k-1}e^{-(x/\lambda )^{k}},&x\geq 0,\\0,&x<0,\end{cases}}}
            .\] 
        \item Gumbel
            \[
            {\displaystyle {\frac {1}{\beta }}e^{-(z+e^{-z})}}
            .\] 
            \[
            {\displaystyle z={\frac {x-\mu }{\beta }}}
            .\] 
        \item Pareto
            \[
            {\displaystyle f_{X}(x)={\begin{cases}{\frac {\alpha x_{\mathrm {m} }^{\alpha }}{x^{\alpha +1}}}&x\geq x_{\mathrm {m} },\\0&x<x_{\mathrm {m} }.\end{cases}}}
            .\] 
    \end{enumerate}
\end{document}
