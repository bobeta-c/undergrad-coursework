\documentclass{article}
\usepackage{amsmath} % For math equations
\usepackage{amsfonts} % For math fonts
\usepackage{amssymb} % For math symbols
\usepackage{enumitem}
\setlist[enumerate,1]{label=\arabic*.}
\setlist[enumerate,2]{label=\alph*.,itemindent=2em}


\title{Homework 1}
\author{Asher Christian 006-150-286}
\date{2024-04-13}

\begin{document}
    \maketitle
    \section{}
       $N$ people, $N_1$ males, $N_0$ females, $T_1$ males above 6 feet, $T_0$ females above 6 feet
       $A$ event person is male, $B$ event person is above 6 feet.
       \begin{enumerate}
           \item Calculations
           \begin{enumerate}
               \item $P(A) = \frac{N_1}{N}$
               \item $P(B) = \frac{T_1+T_0}{N}$
               \item $P(A|B) = \frac{P(A \cap B)}{P(B)} = \frac{\frac{T_1}{N}}{\frac{T_1+T_0}{N}} = \frac{T_1}{T_1+T_0}$
               \item $P(B|A) = \frac{P(B \cap A)}{P(A)} = \frac{\frac{T_1}{N}}{\frac{N_1}{N}} = \frac{T_1}{N_1}$
               \item $P(A \cap B) = \frac{T_1}{N}$
           \end{enumerate}
           \item Verify $P( A \cap B) = P(A)P(B|A) = P(B)P(A|B)$
           \begin{itemize}
            \item This can be verified directly from
              rearranging the conditional Probability formula which has been shown in class
               $P(A|B) = \frac{P(A \cap B)}{P(B)} \rightarrow P(A|B)P(B) = \frac{P(A \cap B)}{P(B)}P(B) = P(A \cap B)$. Because  $P(A \cap B) = P(B \cap A)$ Swapping A for B results in the same equation.
       \item $P(A \cap B) = \frac{T_1}{N} = P(A)P(B|A) = \frac{N_1}{N}\frac{T_1}{N_1} = \frac{T_1}{N} = P(B)P(A|B) = \frac{T_1+T_0}{N}\frac{T_1}{T_1+T_0} = \frac{T_1}{N}$
           \end{itemize}
           \item Verify $P(B) = P(A)P(B|A) + P(A^{c})P(B|A^{c})$.
           \begin{itemize}
               \item  $P(A)P(B|A) + P(A^{c})P(B|A^{c}) = P(A \cap B) + P(A^{c} \cap B) = P(B \cap A) + P(B \cap A^{c}) = P(B)$
           \item $P(B) =  \frac{T_1 + T_0}{N} \rightarrow P(A)P(B|A) + P(A^{c})P(B|A^{c})  = \frac{N_1}{N}\frac{T_1}{N_1} + \frac{N_0}{N}\frac{T_0}{N_0} = \frac{T_1}{N}+\frac{T_0}{N} = \frac{T_1+T_0}{N} $
           \end{itemize}
       \item Verify $P(A|B) = \frac{P(A \cap B)}{P(B)} = \frac{P(A)P(B|A)}{P(A)P(B|A) + P(A^{c})P(B|A^{c})}$
           \begin{itemize}
           \item The first equality was verified in class by the venn diagram. $P(A)P(B|A) = P(A \cap B)$ was proved in question 2. $P(B) = P(A)P(B|A) + P(A^{c})P(B|A^{c})$ was verified directly in question 3 
           \item We know $P(A|B) = \frac{T_1}{T_1+T_0}$ so we show $\frac{P(A)P(B|A)}{P(A)P(B|A)+P(A^{c})P(B|A^{c})} = \frac{\frac{N_1}{N}\frac{T_1}{N_1}}{\frac{N_1}{N}\frac{T_1}{N_1}+\frac{N_0}{N}\frac{T_0}{N_0}} = \frac{T_1}{T_1+T_0}$
           \end{itemize}

       \end{enumerate}
    \section{}
    $X$ and $Y$ are generated from a uniform distribution over [0,1] independently.
    \begin{enumerate}
        \item find $P(X^2+Y^2 \leq 1)$ 
            $\frac{\text{Area of quarter circle with radius one}}{\text{Area of unit square}} = \frac{\frac{\pi}{4}}{1} = \frac{\pi}{4}$
        \item For large $n$ with $m$ random points landing inside the area. Then $\frac{m}{n} \approx  \pi$ so $m \approx n\pi$
    \item Calculate $P(X \geq \frac{1}{2}) \text{and} P(X \geq \frac{1}{2}|X+Y \geq 1)$:\\
        $P(X \geq \frac{1}{2}) = \frac{1}{2}$ \\$P(X \geq \frac{1}{2}|X+Y \geq 1) = \frac{\frac{3}{8}}{\frac{1}{2}} = \frac{3}{4}$
    \item area of $P(A \cap B) = \frac{\text{Area of $A \cap B$}}{\text{Area of square}} = \frac{2}{25} = \frac{2}{5}*\frac{1}{5} = P(A)P(B)$ 
    \end{enumerate}
    \section{}
    In the simplest case all drops can be generalized to a needle dropping in the upper half (by switching our orientation if it lands below)
    with an angle between the horizontal and the upper line. Assuming all orientations are equally likely this is
    equivalent to finding the area in which the distance $D$ between the midpoint and the top is less than $\frac{1}{2} * \sin{\theta}$ with $D \in [0,\frac{h}{2}]$ and $\theta \in [0,\frac{\pi}{2}]$
    this results in \[
        P(\text{needle doesn't hit the edge}) = \frac{\int_0^{\frac{\pi}{2}}\frac{1}{2}\sin{\theta}d\theta}{\int_0^{\frac{\pi}{2}}\frac{1}{2}d\theta}
    .\] Which equals $\frac{2}{\pi}$
    In a Monte-Carlo situation $\frac{m}{n} = P(A) = \frac{2}{\pi} \rightarrow \pi = \frac{2n}{m}$



\end{document}

