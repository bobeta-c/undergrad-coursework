\documentclass{article}
\usepackage{amsmath} % For math equations
\usepackage{amsfonts} % For math fonts
\usepackage{amssymb} % For math symbols
\usepackage{float}
\usepackage{enumitem}
\usepackage{graphicx}
\setlist[enumerate,1]{label=\arabic*.}
\setlist[enumerate,2]{label=\alph*.,itemindent=2em}

\title{HW 1 - 110AH}
\author{Asher Christian 006-150-286}
\date{07.10.24}

\begin{document}
    \maketitle
    \section{Problem 1}
    The order of a finite group = the number of elements in the group. Find all the groups of order 4.
    \begin{enumerate}
        \item $ \mathbb{Z} / 4 \mathbb{Z}$,$+$ = $\{0,1,2,3\}$
            \begin{tabular}{c|c|c|c|c|}
                  & 0 & 1 & 2 & 3 \\ \hline
                0 & 0 & 1 & 2 & 3 \\ \hline
                1 & 1 & 2 & 3 & 0 \\ \hline
                2 & 2 & 3 & 0 & 1 \\ \hline
                3 & 3 & 0 & 1 & 2 \\ \hline
            \end{tabular}
        \item other options = $\{e,a,b,c\}$
            suppose $a \times a \times a = e$. $a \ne a$,  $a \ne e$,  $a = b$ or $a = c$.
            Let $a\times a = b$ then we have  $b\times a = a\times b = e$ and $b\times b = a\times a \times a \times a = a \times e = a$
            \begin{tabular}{c|c|c|c|c|}
                  & e & a & b & c \\ \hline
                e & e & a & b & c \\ \hline
                a & a & b & e &   \\ \hline
                b & b & e & a &   \\ \hline
                c & c &   &   &   \\ \hline
            \end{tabular}\\
            since in any group if $ab = ac \Rightarrow b = c$,  $ac = c \Rightarrow a = e$ a contradiction. Therefore
            it is impossible for any element $g \ne e$ with $g \times g \times g = e$ \\
            Now we have covered cases where $a\times a \times a = e$ and cases where  $a\times b = 0, b \ne a\times a$ 
            The other case is  $a\times a = e$ and $b\times b = e$ for $b \ne a$ \\
            \begin{tabular}{c|c|c|c|c|}
                  & e & a & b & c \\ \hline
                e & e & a & b & c \\ \hline
                a & a & e & c & b \\ \hline
                b & b & c & e & a \\ \hline
                c & c & b & a & e \\ \hline
            \end{tabular}\\    We have $a\times a = b\times b = c\times c = e$
        
    \end{enumerate}

    \section{problem 2}
    Find the greatest common divisor of 2124 and 1024, systematically
    \\
    \begin{align*}
        2124 &= 1024(2)+76\\
        1024 &= 76(13) + 36\\
        76 &= 36(2) + 4\\
        36 &= 4(9) + 0
    \end{align*}
    gcd(2124,1024) = 4
    \section{problem 3}
    Show that the gcd of 111 and 113 is 1 and find $n,m \in \mathbb{Z} \rightarrow 111m + 113n = 1$
    \begin{align*}
        113 &= 111(1) + 2\\
        111 &= 2(55) + 1\\
        2 &= 2(1) + 0
    \end{align*}
    gcd(111,113) = 1
    \begin{align*}
        1 &= 111 - 2(55)\\
          &= 111 - (113-111)(55)\\
          &= 111(56) - 113(55)
    \end{align*}
    $m = 56, n = -55$
    \section{Problem 4}
    Assume (as we shall prove later) that if  $g \in G$ and $n =$ the smallest positive integer
    such that $g^{n} = e$ then $n|\text{ord(G)}$ where ord(G) = the order of G = number of elements in G.
    Prove:
    \begin{enumerate}
        \item (a) If $G$ is a finite group, $g \in G$ then $\exists N_g>0 \rightarrow g^{N_g} = e$ 
            so there is smallest such $N_g$ notation ord(g)
        \item (b) use the given assumption to show that if ord( $G$ ) = $p$, $p$ prime
            then $G$ is the same group as $ \mathbb{Z}/ \mathbb{Z}_p$ 
        \item Proof (a)\\
            assume false for contradiction. Then  consider $n = $ ord(G) = \# of elements in G and the corresponding
            and the list of n+1 elements $g, g^2, ... , g^{n}, g^{n+1}$. The list must then contain
            no duplicates for if \\$g^{i} = g^{j}$ with $ i,j \in [1,n+1] i < j$ \\
        Then $g^{j-i}g^{i} = eg^{i} \rightarrow g^{j-i} = e$ a contradiction. 
            However there are n+1 elements in the list and only n elements in the group
            so by pigeonhole principle 2 elements must be the same yielding a contradiction and proving our assumption
            that there is no $N_g$ false. Therefore one of the $g^{2}, g^{3}, ..., g^{n+1}$ is g and 
            take the power that is associated with the least satisfying that property to be $N_g$
        \item Proof (b)
            By the assumption any $g \in G$ with $g^{n} = e$ take n minimal with this quality  n $|$ ord(G)  and $n \le$ ord(G)
            However ord(G) is prime so n = 1 or n = p. if n = 1 then $g = e$ so consider n = p. then the group is spanned by
            $\{e, g, g^{2}, ..., g^{p-1}\}$ Take the power to represent the element so the set becomes
            $\{0, 1, 2, ..., p-1\}$ and multiplying elements means adding powers just like adding numbers in the  $ \mathbb{Z}/ \mathbb{Z}_p$ 
            Set so they are isomorphic.
    \end{enumerate}
    \section{Problem 5}
    Let  $S_3 =$ the group of 1-1 functions from $1,2,3$ to itself.
    \begin{enumerate}
        \item (a) show $S_3$ is a group when $\times$ = composition of functions (on the right)
            $(f \times g)(x) = g(f(x))$ 
        \item (b) What is the order of $S_3$?
        \item (c) Is $S_3$ the same group (except for notation) as $ \mathbb{Z} / $ ord($S_3$)$ \mathbb{Z}$?
        \item Proof (a).
            Assuming all functions are 1-1 onto otherwise inverse functions would be impossible.
            $e(f(x)) = f(e(x)) = f(x)$ where e is the function that maps by the following ordered pairs  $(1,1), (2,2), (3,3)$ 
            for any function (1, a), (2, b), (3,c) $a,b,c \in \{1,2,3\} a \ne b \ne c$ the corresponding function (a,1), (b,2), (c,3) is 
            the inverse. for any series of functions a,b,c a(bc) = c(b(a(x))) = (ab)c. Therefor the set is a group under composition
        \item (b)
            there are 6 elements in the group - the first element can be mapped to 3 elements, the second to 2 and the third to 1, multiply together to get 6 total options
        \item proof (c).
            if $S_3$ is the same as $ \mathbb{Z} / \mathbb{Z}_{ord(S_e)}$ then for each element  $g \in S_3 \rightarrow g^{6} = e$
            and one element spans the entire set. Let (a,b,c) denote a function that maps 1-a, 2-b, 3-c
            \[
             \begin{cases}
                (1,2,3) & g=e\\
                (2,1,3) & g^2=e\\
                (3,2,1) & g^2=e\\
                (1,3,2) & g^2=e\\
                (2,3,1) & g^3=e\\
                (3,1,2) & g^3=e
            \end{cases}
            .\] 
            There is no element that spans the set so the set is not isomorphic to $ \mathbb{Z} / \mathbb{Z}_{ord(S_3)}$

    \end{enumerate}
    \section{Problem 6}
    Suppose $N \in \mathbb{Z}^{+}$. Prove:\\
    \begin{enumerate}
        \item
            There exists only finitely many finite groups $G \rightarrow ord(G) = N$.(Regarding $G_1$ and $G_2$ as the same if they are "isomorphic"\\
        \item 
            Proof\\
            Consider $g_1,g_2,g_3 \in G$ unique elements. if $g_1g_2=g_1g_3 \Rightarrow g_1^{-1}g_1g_2=g_1^{^-1}g_1g_2 \Rightarrow g_2=g_3$.
            Take the ordering $a_m = (x_1,x_2,x_3,...x_{n-1}), m<n, x_i \in G, 1 \le i < n$
            For any group $a_1$ has $(n-1)!$ possible combinations and every subsequent $a_i$ has $(n-i)!$ options. So the total combinations is
            equal to $\sum_{i=1}^{n-1}(n-i)!$ $\ne \infty$
            

    \end{enumerate}
    \section{Problem 7}
    Look at $(N+1)^{3}-N^{3} = 3N^2+3N+1$ and sum the LHS from $N=1 to N=n$ use the fact you know
    $\sum_{1}^{n}N$ and $\sum_{1}^{n}1$ to figure out what $\sum_{1}^{n}N^2$ is!
    \\
    We know $\sum_{N=1}^{n}1 = N$ and $\sum_{N=1}^{n}N = \frac{n(n+1)}{2}$ \\
    Assume $\sum_{N=1}^{n}N^{3}$ follows a polynomial of degree 3
    \begin{align*}
        a(n+1)^{3}+b(n+1)^{2}+c(n+1)+d &= an^{3}+b^2+cn+d +(n+1)^2 \\
        an^{3}+3an^2+3an+a+bn^2+2bn+b+cn+c+d &=\\
        an^{3}+(3a+b)n^2+(3a+2b+c)n+(a+b+c+d) &= an^{3} + (b+1)n^2+(c+2)n+(d+1)\\
    \end{align*}
    \[
    a=\frac{1}{3}, b=\frac{1}{2},c=\frac{1}{6}
    .\] 
    \[
    \sum_{N=1}^{n}N^2=\frac{1}{3}n^{3}+\frac{1}{2}n^2+\frac{1}{6}n
    .\] 
    \[
    \sum_{N=1}^{n}3N^2+3N+1 = n^{3}+\frac{3}{2}n^2+\frac{1}{2}n+\frac{3}{2}n^2+\frac{3}{2}n+n= n^3+3n^2+3n
    .\] 
    Note - I was not very smart and I did not realize how plainly it was put. Alow me to try again.\\

    Consider the sequence $\sum_{N=1}^{n}N^{3}$, $1,2^{3}, 3^{3}, 4^{3},...,n^{3}$. and
    $\sum_{N=1}^{n}(N+1)^{3}$ = $2^{3},3^{3},4^{3},5^{3},...,n^{3},(n+1)^{3}$. Subtracting the larger sum from the smaller we get.\\
    $(n+1)^{3}-1 = \sum_{N=1}^{n}(N+1)^{3}-N^{3}$ \\
    \begin{align*}
        (n+1)^{3}-1 &= 3\sum_{N=1}^{n}N^2 + \frac{3n(n+1)}{2} + n\\
        n^{3}+3n^2+3n+1-1-n-\frac{3}{2}n^2-\frac{3}{2}n&=3\sum_{N=1}^{n}N^2\\
        \frac{n^{3}}{3}+\frac{1}{2}n^2+\frac{1}{6}n&=\sum_{N=1}^{n}N^2
    \end{align*}
    The same result obtained before.


    \section{Problem 8}
    Can you do this process on prob 4 for higher powers (inductively on the power)? How does it work for 3rd powers?\\
    Yes.
    Assume the formula is known for $\sum_{N=1}^{n}N^{a}$ for some $a \in \mathbb{N}$. To determine $ \sum_{N=1}^{n}N^{a+1}$ simply
    consider $\sum_{N=1}^{n}(N+1)^{a+2}-N^{a+2} = (n+1)^{a+2}-1 =$ sum of powers less $a+2$. Every power less than $a+1$ can be written in temrs of $n$
    and the remaining $\sum_{N=1}^{n}N^{a+1}$ can be expressed in terms of powers of n which is the end goal.



\end{document}
