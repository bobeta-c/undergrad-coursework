\documentclass{article}
\usepackage{amsmath} % For math equations
\usepackage{amsfonts} % For math fonts
\usepackage{amssymb} % For math symbols
\usepackage{float}
\usepackage{enumitem}
\usepackage{graphicx}
\setlist[enumerate,1]{label=\arabic*.}
\setlist[enumerate,2]{label=\alph*.,itemindent=2em}

\title{HW 1 - 110AH}
\author{Asher Christian 006-150-286}
\date{07.10.24}

\begin{document}
    \maketitle
    \section{}
    For $n \ge 2$, let $B_n : \{1,2,...n\} \rightarrow \{1,...,n\}$
    be defined by  $B_k(n) = k+1-n$ 
    ($B_n$ is the permutation that lists $\{1,...,n\}$ in reverse order).
    Find $sgn(B_n)$\\
    Starting with $B_n(1)$, it appears at the end of the list of elements so the number it contributes to the sum is $0$.
    For each $B_n(k)$ there are $k-1$ elements ahead of it with value less than $B_n(k)$.
    This sum can be written as
     \[
    \sum_{i=1}^{n}(i-1) = \frac{n^2-n}{2}
    .\] 
    Therefore $sgn(B_n) = (-1)^{\frac{n^2-n}{2}}$
    \section{}
    Show that $\pi_{ij} : \{1,...,n\} \rightarrow \{1,...,n\}$ with $i,j \in \{1,...,n\}$ defined by
    \begin{align*}
        \pi_{ij}(k) &= k, \; k \ne i,j, i\ne j\\
        \pi_{ij}(i) &= j\\
        \pi_{ij}(j) &= i\\
    \end{align*}
    has $sgn(\pi_{ij}) = -1$\\
    consider $\pi_{ij}(k), \; i < j, \; k < i, \text{or} \; k > j$. Clearly every element above $\pi_{ij}(k)$ is greater than $k$ so it contributes none to the sum.
    The same goes for $\pi_{ij}(i)$. Then for each $i < n < j$  the sum of numbers greater than the permutation $\pi_{ij}(n) = 1$. Because the only number less than $n$ above $\pi_{ij}(n)$ is $i$.
    There are $j-i-1$ of these $n$ bringing the total count to $j-i-1$.
    for $\pi_{ij}(j)$ There are all of the numbers plus one in between so $j-i$ numbers that are less than $j$. Adding this to the count we get $2i-2j-1$.
    This number is always odd because it is one less than 2 times an integer and so the $sgn(\pi_ij) = -1$ for all $i,j$

    \section{}
    Is every element of $S_n$ (symmetric group on $n$ elements) a product (composition) of 
    interchanges $\pi_{ij}$ (as in previous problem)?
    \\
    Starting with the identity $I$ and any bijection $S_i$. pick $j_1$ such that
    $S_i(j_1) = 1$. Then compose $\pi_{1,j}I$ to get $I_1$. now $I_1(j) =  1$. Proceed inductively on  $n$ and for each $n$ first find $k_n$ such that $S_i(k_n) = n$ and the $m_n$ such that $I_{n-1}(m_n) = n$ compose $\pi_{m_n,k_n}I_{n-1}$. 
    All $k_i$ are unique because the permutation is a bijection and all $m_n$ are unique because the previous $m_i, i < n$ point to other elements.
    Thus with this technique every element of $S_n$ can be constructed as a composition of $\pi_{ij}$
    
    \section{}
    Explain why $A_n = \{S_{ \sigma } \in S_n: sgn(\sigma ) = 1$\}. Is a normal subgroup by showing that $\pi  \rightarrow sgn(\pi )$ is a homomorphism of $S_n$ onto the group $\{-1,1\}$, operator multiplication
    \\
    for any two permutations $\sigma_1, \sigma_2$. $sgn(\sigma_1\sigma_2) = sgn(\sigma_1)*sgn(\sigma_2)$.
    This follows from the fact that any adjacent interchange applied to $\sigma $ =  $\pi_{i,i+1}\sigma $ flips the sign by $-1$ and any $\pi_{ij}$ can be written
    as
    \[    
    \pi_{ij}=\pi_{i,i+1}\pi_{i+1,i+2}...\pi_{i+n,j}\pi_{i+n,i+n-1}...\pi_{i,i+1}
    .\]
    As a sanity check it is quickly verified that the above expression has an odd number of elements.\\
    So from the previous question any permutation is a product of adjacent interchanges and therefore its sign
    is known by the amount of adjacent interchanges it has multiplied by the sign of the identity which is 1.
    \\
    Therefore for any $\pi_1, \pi_2 \in S_n$ $sgn(\pi_1\pi_2) = sgn(\pi_1)sgn(\pi_2)$ and the range of  $sgn$ is $-1,1$ so there is a homomorphism.
    For any subset $A_n$ of $S_n$ and any $a_1,a_2 \in A_n$ $sgn(a_1) = sgn(a_2) = 1$ and $sgn(a_1a_2) = 1*1 = 1$ so the result of any two elements
    composed with themselves is a member of the subgroup and thus the subgroup is normal.
    
    \section{}
    Check explicitly that $S_3$ has one subgroup of order 3 which is normal and three subgroups of order 2,
    none of which is normal.
    $S_3$ has 6 elements defined
    \begin{align*}
        (1,2,3) &= e\\
        (2,1,3) &= T_{1,2}\\
        (1,3,2) &= T_{2,3}\\
        (3,2,1) &= T_{1,3}\\
        (3,1,2) &= g_3\\
        (2,3,1) &= g_2
    \end{align*}
    Notably their inverses and squares are:
    \begin{align*}
        e^{-1} &= e\\
        T_{1,2}^{-1} &= T_{1,2}\\
        T_{2,3}^{-1} &= T_{2,3}\\
        T_{1,3}^{-1} &= T_{1,3}\\
        g_3^{-1} &= g_2\\
        g_2^{-1} &= g_3\\
        g_3^2 &= g_2\\
        g_2^2 &= g_3
    \end{align*}
    From this the following cyclic subspaces come easily: 
    $\{e,g_3,g_2\},\{e,T_{1,2}\},\{e,T_{2,3}\},\{e,T_{1,3}\}$.
    Aside from the one listed, any subgroup of order 3 would require 2 distinct  $T$'s However 
    the combination of  $T_iT_j$ with $i\ne j$ is not in the set $\{e,T_i,T_j\}$ Because it is necessarily not equal to the identity.
    Thus the cyclic group is the only subgroup of order 3.
    It is easy to check that it is normal because $T_{1,2}g_3T_{1,2} = g_2$, $T_{1,3}g_3T_{1,3} = g_2$, $T_{2,3}g_3T_{2,3} = g_2$.
    Similarly the same holds in the previous equations if every instance of $g_2$ is swapped with $g_3$ and vice versa from the equation
    $T_ig_3T_iT_ig_3T_i = T_ig_2T_i = g_2^2 = g_3$.\\
Thus $\{e,g_3,g_2\}$ is a normal subgroup of order 3.
    \\
    For the subgroups of order 2 there are clearly no others because they would have to be cyclic.
    They are also not normal as shown early that $T_iT_j \ne T_i \; \text{or} \; e$.
    All subgroups must be order 3 or 2 so we have exhaustively checked all subgroups and determined that the one of order 3 is normal
    and the 3 of order 2 are not.

    \section{}
    \begin{enumerate}
    \item 
        Prove without computing the compositions that $\pi_{12}$ and  $\pi_{23}$ do not commute
    by noting that if they did commute then $\{e, \pi_{12}, \pi_{23}, \pi_{12}\pi_{23}\}$ would be a subgroup of $S_3$ of order $4$ 
    \\
    we know that $\pi_{12}^{-1} = \pi_{12}$ and $\pi_{23}^{-1} = \pi_{23}$
    This information shows that the set $e, \pi_{12}, \pi_{23}, \pi_{12}\pi_{23}$ would be a subgroup
    for $\pi_{12}\pi_{23} = \pi_{23}\pi_{12}$ and the inverse would be itself for $\pi_{12}\pi_{23}\pi_{23}\pi_{12} = \pi_{12}e\pi_{12} = \pi_{12}\pi_{12} = e$
    This is a violation because $4 \nmid 6$  

    \item
    Compare $\pi_{12}\pi_{23}$ and $\pi_{23}\pi_{12}$ to see that they are different.
    $\pi_{12}\pi_{23} = (2,3,1)$ and $\pi_{23}\pi_{12} = (3,1,2)$
    they are clearly different.

 
    \end{enumerate}

    \section{}
    If $n \ge 3$ and $k < n$, consider the $k$-cycle's the permutations of  $(1,...,n)$ 
    generated by looking at $1,...,k$ and moving it to the right $1\rightarrow 2$ $2\rightarrow 3$  $k-1 \rightarrow k$ $k \rightarrow 1$ 
    \begin{enumerate}
        \item What is $sgn(\sigma )$?
            \\
            By counting we have $k$ has $k-1$ elements ahead of it that are less than $k$ and all other numbers are the less than all numbers ahead of them
            so $sgn(\sigma ) = (-1)^{k-1}$
        \item What is the order of $\sigma$?
            There are $k$ elements corresponding to the possible values of $k$.
    \end{enumerate}
    
    \section{}
    Think about $k$-cycles on every subset of $k$ elements in $\{1,...,n\}$ 
    \begin{enumerate}
        \item is every element of $S_3$ a product of disjoint cycles? (k arbitrary)\\
            Yes. 
            $e, T_{1,2}, T_{1,3}, T_{2,3}$ are all a singular cycle on a subset of ${1,...,3}$
            \\
            $g_2$ and $g_3$ are both cycles on the entire set.
        \item $S_4, S_n$?\\
            Yes.
            There are $n!$ elements of $S_n$. Consider any $\sigma \in S_n$ and the list created
        by $(1, \sigma(1), \sigma^2(1),...,\sigma^{n!}(1))$. This list has $n!+1$ elements
        so it must contain one element twice and because $\sigma $ is a bijection 1 must be repeated.
        Consider the first occurance $\sigma^{k+1}(1) = 1$ then $(1, \sigma(1), \sigma^2(1), ..., \sigma^k(1))$ has only distinct elements.
        This set also necessarily has at least one element being 1 so $\{1,...,n\} \setminus \{1, \sigma(1),...,\sigma^{k}(1)\}$ is of cardinality strictly
        less than $n!$. Consider any $i$ in this newly created set and the list $(i, \sigma(i), ..., \sigma^{k_i}(i))$ with $\sigma^{k_i+1}(i) = i$
        This list is entirely contained in $\{1,...,n\} \setminus \{1,\sigma(1),...,\sigma^{k}(1)\}$ because cycles of disjoint elements are disjoint by the bijectivity of $\sigma $.
        Proceed inductively on $i$ choosing a new $i$ and continually subtracting elements from the starting set $1,...,n$. Each iteration
        generates a distinct cycle and each iteration strictly decreases the number of elements remaining. Therefore it terminates when the last element(s) are put into cycles.
        

    \end{enumerate}

    
    
\end{document}
