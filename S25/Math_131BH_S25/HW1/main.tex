\documentclass[11pt]{article}

\newcommand{\cnum}{Math 131BH}
\newcommand{\ced}{Spring 2025}
\newcommand{\ctitle}[4]{\title{\vspace{-0.5in}\cnum, \ced\\Homework Set #1: #2}\author{\vspace{-0.35in}\\#3, #4}}
\usepackage{enumitem}
\usepackage[usenames,dvipsnames,svgnames,table,hyperref]{xcolor}
\usepackage{amsmath, amsfonts}
\usepackage{graphicx} % Required for inserting images
\usepackage{float}
\usepackage{listings}
\usepackage{xcolor}
\usepackage{hyperref}
\usepackage{comment}

\renewcommand*{\theenumi}{\alph{enumi}}
\renewcommand*\labelenumi{(\theenumi)}
\renewcommand*{\theenumii}{\roman{enumii}}
\renewcommand*\labelenumii{\theenumii.}

\definecolor{codegreen}{rgb}{0,0.6,0}
\definecolor{codegray}{rgb}{0.5,0.5,0.5}
\definecolor{codepurple}{rgb}{0.58,0,0.82}
\definecolor{backcolour}{rgb}{0.95,0.95,0.92}
\definecolor{header}{HTML}{205459}
\definecolor{solution}{HTML}{32848c}

\newcommand{\solution}[1]{{{\textcolor{header}{Solution:} \textcolor{solution}{#1}}}}
\setlist[enumerate]{label=(\alph*)}

\lstdefinestyle{mystyle}{
    backgroundcolor=\color{backcolour},
    commentstyle=\color{codegreen},
    keywordstyle=\color{magenta},
    numberstyle=\tiny\color{codegray},
    stringstyle=\color{codepurple},
    basicstyle=\ttfamily\footnotesize,
    breakatwhitespace=false,
    breaklines=true,
    captionpos=b,
    keepspaces=true,
    numbers=left,
    numbersep=5pt,
    showspaces=false,
    showstringspaces=false,
    showtabs=false,
    tabsize=2
}


\begin{document}
\ctitle{\#0}{}{Asher Christian}{006-150-286}
\date{}
\maketitle
\section{Lemma for E1 and E5}
I introduce this lemma because I will use it in multiple exercises, so I want to keep it here for clarity.
\\
If $(X,d)$ is a metric space with the following property: for any $(x_n)_n \subset X$ sequence and any $\epsilon > 0$
there exists some subsequence $(x_{n_k})_k \subset X$ and some $N > 0 \in \mathbb{N}$ such that for any $n,m > N$
 \[
d(x_n,x_m) < \epsilon
.\] 
then there exists a subsequence $(x_{n_l})_l$ of $(x_n)_n$  that is cauchy.\\
\\
Proof:\\
For any $\epsilon > 0$ and the sequence $(x_{n_k})_k, N$ defined as above we can rewrite the sequence disregarding the first $N$ terms
creating a new sequence $(x_{n_{k'}})_{k'}$
such that for any $p,q \in (x_{n_{k'}})_{k'}$, $d(p,q) < \epsilon$, so we can implicitly redefine all future sequences in this way.\\
Denote the subsequence of $(x_n)_n$ generated by $\epsilon = \frac{1}{K}$ as $(x^{K}_n)_n$
Using all of the above I generate my cauchy subsequence $(y_n)_n$ as follows:\\
pick $y_1$ to be the first element of $(x^{1}_n)_n, \; x_1^{1}$ then consider the sequence $(x_n^{1})_{n=2}^{\infty}$ this sequence is infinite and is contained
in $X$ so by the statement there exists a subsequence of $(x_n^{1})_{n=2}^{\infty}$ such that it satisfies the above criterion with $\epsilon = \frac{1}{2}$.
Pick $y_2$ to be the first element of this new subsequence $(x^{2}_n)_n$ and create a new subsequence of this sequence minus its first element with $\epsilon = \frac{1}{3}$.
Repeating inductively we define $y_{k}$ to be the first element of the sequence $(x^{k}_n)_n$ where $(x^{k}_n)_n$ is defined as the $\epsilon = \frac{1}{k}$ subsequence of
$(x^{k-1}_n)_{n=2}^{\infty}$. This sequence is well defined because each element is determined by a finite amount of subsequences. Additionally this sequence is a proper
subsequence of $(x_n)_n$ since every element of $(y_n)_n$ is an element of $(x_n)_n$ and the indices are strictly increasing. lastly $(y_n)_n$ is cauchy.
this is because for any indices $i,j, i<j$, $d(y_i,y_j) \le \frac{1}{i}$ since $y_j$ and $y_i$ were both elements of $(x^{i}_n)_n$ and are therefore within $\frac{1}{i}$ from eachother.
Thus for any $\epsilon >0$ picking $N$ such that $\frac{1}{N} < \epsilon$ we get that if $n,m > N$ that $d(y_n,y_m) < \epsilon$ and our sequence is cauchy.
\section{Exercise 1}
\emph{
    Show that in the metric space $(l^2,d_2),$ the set
    $E := \{x = (x_n): \sum_{n=1}^{\infty}n|x_n|^2 \le 1\}$ is compact.
}
\solution{
Assuming $l^2$ is defined over $ \mathbb{R}$, then $l^2$ is complete, in particular $E$ is complete since it is closed for consider any sequence
\[
    (x_n)_n \subset E \hspace{1cm} \lim_{n\to \infty}x_n \rightarrow x \in l^2
.\] 
since $\sum_{i=1}^{M}i|x_n(i)|^2 \le 1 \forall M \forall n$. we have
\[
    \lim_{n\to \infty} \sum_{i=1}^{M}i|x_n(i)|^2 \le 1 \hspace{1cm} \forall m
.\] 
which implies that
\[
\lim_{M\to \infty}\lim_{N\to \infty}\sum_{i=1}^{M}i|x_n(i)|^2 \le 1
.\] 
thus
\[
\sum_{i=1}^{\infty}i|x(i)|^2 \le 1
.\] 
and $x \in E$.\\
Additionally note that for any $x \in E$, $-\frac{1}{\sqrt{i}} \le x(i) \le \frac{1}{\sqrt{i}}$
\\
Now consider any sequence
\[
    (x_n)_n \subset E, x_n \in E \; \forall n
.\] 
construct our subsequence $(s_n)_n \subset (x_n)_n$ as follows:\\
Consider  the sequence of real numbers created by the first element in each element of our sequence
\[
    (x_n(1))_n
.\] 
Since $-1 \le x_n(1) \le 1$ since $[-1,1] \subset \mathbb{R}$ is compact and thus sequentially compact there exists a subsequence
\[
    (x_{j}^{1})_j \subset (x_n)_n
.\] 
such that the first index of elements in the sequence converge to some element in $[-1,1]$. Let $s_1 = x^{1}_1$.
create $s_n$ as follows.\\
First construct a subsequence
\[
    (x_j^{n})_j \subset (x_j^{n-1})_j
.\] 
such that the sequence of real numbers created by the $n$th element in each element of our sequence
\[
    (x_j^{n}(n))_j 
.\] 
converges to some element in $[-\frac{1}{\sqrt{n}},\frac{1}{\sqrt{n}}]$. Then 
\[
s_n = x^{n}_n
.\] 
In this way we define the sequence
\[
    (s_n)_n
.\] 
such that for any $j \in \mathbb{N}$,
\[
    (s_n(j))_n
.\] 
converges since the tail of the sequence after the first $j$ elements is a cauchy sequence in that index.
Additionally all the indices of $(s_n)$ are strictly increasing because we chose the $n$th element of the $n$th series.
It suffices to show that $(s_n)_n$ is cauchy, that for any $\epsilon > 0$ there exists $N >0$ such that if $n,m > 0$ 
\[
\sum_{i=1}^{\infty}|s_n(i)-s_m(i)|^2 \le \epsilon
.\] 
first for any index $M$ we have
\[
\sum_{i=M+1}^{\infty}|s_n(i) - s_m(i)|^2 \le 2\sum_{i=M+1}^{\infty}|s_n(i)|^2 + 2\sum_{i=M+1}^{\infty}|s_m(i)|^2
.\] 
but
\[
\sum_{i=M+1}^{\infty}|s_n(i)|^2 \le \frac{2}{M}\sum_{i=M+1}^{\infty}i|s_n(i)|^2 \le \frac{2}{M}
.\] 
since $s_n$ and $s_m$ are in $E$ so their weighted sums are bounded by $1$ thus
\[
\sum_{i=M+1}^{\infty}|s_n(i)-s_m(i)|^2 \le \frac{4}{M}
.\] 
Since for any $M$ $s_n$ is cauchy in the first $M$ elements, for any $\epsilon$ we can find an $N_i$ for each index
$i < M$ such that $(s_n(i))_i$ is cauchy within epsilon for  $n,m > N_i$. Taking $N = \max\{N_1,...,N_M\}$ we can get the first $M$ indices
all cauchy within any $\epsilon$ each and picking that $\epsilon$ based on $M$ we can get the sum of their differences bounded for any new $\epsilon$.\\
picking $M > \frac{8}{\epsilon}$ and $N$ such that differences in the first $M$ indices are less than $\sqrt{\frac{\epsilon}{2M}}$ we get for any $n,m > N$
\[
\sum_{i=1}^{M}|s_n(i)-s_m(i)|^2 \le \sum_{i=1}^{M}|\sqrt{\frac{\epsilon}{2M}}|^2 = \sum_{i=1}^{M}\frac{\epsilon}{2M} = \frac{\epsilon}{2}
.\] 
and from the previous argument
\[
\sum_{i=M+1}^{\infty}|s_n(i)-s_m(i)|^2 \le \frac{2}{M} \le  \frac{\epsilon}{2}
.\] 
and so
\[
\sum_{i=1}^{\infty}|s_n(i)-s_m(i)|^2 \le \epsilon
.\] 
thus our sequence
\[
    (s_n)_n
.\] 
is cauchy in a complete metric space and thus converges to some element of $E$. So for every sequence $(x_n)_n \subset E$ we can find a convergent subsequence
$(s_n)_n \subset (x_n)_n$ and $E$ is sequentially compact which we will show in the next exercise implies that $E$ is compact.
}
\section{Exercise 2}
\emph{
    \begin{enumerate}
        \item Show that if $(K,d)$ is sequentially compact, then $(K,d)$ has countable base
        \item Using $(a)$ show that every open cover of $K$ has a countable subcover.
    \end{enumerate}
}
\solution{
    \begin{enumerate}
        \item for any $r \in \mathbb{R} > 0$ and set
            \[
                \{V^r_\alpha\} := \{N_{r}(x) : x \in K\}
            .\] 
            there exist finitely many $\{V^r_{\alpha_i}\}_{i=1}^{n_r}$ that cover $K$, for if that were not the case
            then we could construct a sequence of points $(x_n)_n$ such that $x_0$ is in $V^r_{x_0}$ and
            $x_1 \in K \setminus {V^r_{x_0}}$, $x_2 \in K \setminus \{V^r_{x_0} \cup V^r_{x_1}\}$, $x_n \in K \setminus \{\bigcup_{i=0}^{n-1}V^r_{x_i}\}$
        and each of these sets is nonempty since if for some $n$ , $K \setminus \bigcup_{i=0}^{n-1}V^r_{x_i}$ were empty, then $\bigcup_{i=0}^{n-1}V^r_{x_i} = K$ is a finite
        cover of  $K$. and clearly there is no convergent subsequence  $(x_{n_k})_k$ of  $(x_n)$ since for any $n,m \in \mathbb{N}, n\ne m, d(x_n,x_m) \ge r$.\\
        Since this holds for all $r > 0$ picking $r_n = \frac{1}{n}$ we have the sets
        \[
            \{\{V^{r_i}_{\alpha_j}\}_{j=1}^{n_{r_i}}\}_{i=1}^{\infty}
        .\] 
        form a countable base for $K$. This is because for any $x \in K, G \subset K$ such that $x \in G$ and $G$ open in $K$, there exists some $m > 0$ such that $N_m(x) \subset G$. pick $n \in \mathbb{N} \rightarrow \frac{1}{n} < \frac{m}{2}  $, then there exists some 
        $V_\alpha^{r_n}$ in the base that contains $x$ since each "layer" of the base where the radius is held constant constitutes an open cover of $K$. Additionally
         $V_\alpha^{r_n} \subset N_m(x) \subset G$ since if $l$ is the point in $K$ that $V_\alpha^{r_n}$ is centered around for any $p \in V_\alpha^{r_n}$, $d(p,x) \le d(p,l) + d(l,x) < \frac{1}{n} + \frac{1}{n} \le \frac{1}{m}$ implies $p \in N_m(x)$
         So we have shown for any point in  $K$ and any neighborhood of that point there exists a "smaller" neighborhood of that point in a countable base that we constructed.
     \item Consider any open cover $\{G_\alpha\}_{\alpha\in A}$ of $K$ and a countable base $\{V^{r}_i\}_{i,r \in \mathbb{N}, \mathbb{Q}}$. Then for each $p \in K$,  $p \in G_\alpha$ for some $\alpha \in A$ and there exists some $V_i^{r}$ for some $i,r$ such that $p \in V^{r}_i$ and $V^{r}_i \subset G_\alpha$.
         Proceeding in this way, we get a subset of $\{V_i^{r}\}_{i\in \mathbb{N}, r \in \mathbb{Q}}$ such that each $p$ is in some $V_i^{r}$ and each $V_i^{r}$ is completely contained in some $G_\alpha$.
         Taking the set of $G_\alpha$ corresponding to this subset of $V_i$ we get a countable cover of $K$. Thus each open cover can be reduced to a countable open cover.
    \end{enumerate}
}
\section{Exercise 3}
\emph{
    Show that $K$ is sequentially compact if and only if every infinite subset of $K$ has a limit
    point in $K$.\\
}
\solution{
    For the first direction (1) sequentially compact $\implies$ (2) every infinite subset of $K$ 
    has a limit point in $K$ :\\
    assume for contradiction that $(1)$ holds but not $(2)$ then there exists some infinite subset $S \subset K$ with no limit point in $K$.
    generate a sequence $(x_n)_n$ of unique points in $S$ which is possible since there are infinitely many of them. Then clearly $(x_n)_n$ has
    no convergent subsequence since if $(x_{n_k})_k \rightarrow x \in K$, that would imply $x$ is a limit point in $K$. This contradicts $K$ being sequentially compact and so $(1) \implies (2)$.
    \\
    To prove $(2) \implies (1)$\\
    For any sequence  $(x_n)_n$ consider the set of all points $S := \{x_1,x_2,...\}$ then if $S$ is finite, there must be some $x \in S$ that occurs infinitely many points in the sequence $(x_n)_n$ and thus
    picking this element infinitely many times generates a convergent subsequence. If instead $S$ is infinite, then by $(2)$ there must exist a limit point $p$ of $S$ in $K$. This implies that each ball $N_r(p), r \in \{1,\frac{1}{2},\frac{1}{3},...\}$ contains infinitely
    many points in $S$. Picking a unique point in each of the consecutive balls while ensuring strictly increasing indices creates a convergent subsequence. Thus for any sequence $(x_n)_n \subset K$, there exists a convergent subsequence and $K$ is sequentially compact.
}

\section{Exercise 4}
\emph{
    Let $K,E \subset X$ and $K$ is compact and $E$ is closed in $(X,d)$.
    \begin{enumerate}
        \item If $K \cap E = \emptyset$  then show that there is a constant $c > 0$ such that
            \[
                d(x,E) := \inf\{d(x,y),y\in E\} \ge c \hspace{1cm} \text{for all $x \in K$ }
            .\] 
        \item Is $(a)$ true if $K$ is only closed?  
    \end{enumerate}
}
\solution{
    \begin{enumerate}
        \item since $d(x,E)$ is bounded from below $\alpha = \inf_{x\in K}(d(x,E))$ exists.
            It is our goal to show that $\alpha > 0$. Construct a sequence of points $(x_n)_n$ 
            in $K$ such that $\lim_{n\to \infty}d(x_n,E) \rightarrow \alpha$, then by compactness implies sequential compactness,
            there exists a subsequence $(x_{n_k})_k$ such that $\lim_{n\to \infty}x_{n_k} \rightarrow x_\infty \in K$. Indeed
            $d(x_\infty,E) = \alpha$ but $d(x_\infty,E) > 0$ since $E$ is closed and $E^{c} \supset K$ is open which implies
            that there exists some $r > 0$ such that $N_r(x_\infty) \subset E^{c}$ and so $d(x_\infty,E) \ge r > 0$.
            Thus $\alpha \ge r > 0$ shows that there exists a lower bound on $d(x,E)$ for all $x \in K$.
        \item No this is not true for consider  $X = [1,\infty) \times [0,\infty) = X \times Y$ and $E := \{(x,y): y \ge \frac{1}{x}\}$, $K := \{(x,y): y \le 0\}$ 
            Both $E$ and $K$ are closed but $\inf_{x\in K}d(x,E) = 0$ which can be seen by the line $y = \frac{1}{x}$ approaching $0$, however the two sets never intersect since $\frac{1}{x} > 0$ for all $x$.
    \end{enumerate}
}
\section{Exercise 5}
\emph{
    Let $A$ be a subset of a complete metric space. Assume that for all
    $\epsilon > 0$ there exists a compact subset $A_\epsilon$ so that For any
    $x \in A, d(x,A_\epsilon) < \epsilon$ show that $\overline{A}$ is compact.\\
}
\solution{
    Since we have proved Sequential Compactness $\implies$ Compactness, it suffices to show that
    $\overline{A}$ is sequentially compact. Since $A$ is a subset of a complete metric space $\overline{A}$ is complete so it suffices
    even further to show that every sequence in $\overline{A}$ has a cauchy subsequence.\\
    Consider any sequence $(x_n)_n \subset \overline{A}$. Then for any $\epsilon > 0$ there exists some compact subset $A_{\frac{\epsilon}{5}}$ such that for any $x \in \overline{A}, d(x,A_{\frac{\epsilon}{5}}) < \frac{\epsilon}{5}$. This is because each $x_n$ is either a point in $A$ or a limit point, if it is a point
    it holds by definition, if it is a limit point it is within $\frac{\epsilon}{10}$ from a point in $A$ which is in turn $\frac{\epsilon}{10}$ from $A_\frac{\epsilon}{10}$ so we can always find points within $\epsilon$ from the closure of $A$.
    Thus we can find for each $x_n \in (x_n)_n$ some $p_n$ such that $p_n \in A_{\frac{\epsilon}{5}}$ and $d(p_n,x_n) < \frac{\epsilon}{4}$.
    Now consider the sequence $(p_n)_n$. since $A_{\frac{\epsilon}{5}}$ is compact and thus sequentially compact, there exists some subsequence
    $(p_{n_k})_k$ such that $(p_{n_k})_k$ is cauchy. This implies further that for some $N > 0 \in \mathbb{N}, \;\; \forall i,j > N, d(p_{n_i},p_{n_j}) < \frac{\epsilon}{4}$.
    However this further implies that 
    \[
        d(x_{n_i}, x_{n_j}) \le d(x_{n_i},p_{n_i}) + d(p_{n_i},x_{n_j}) \le d(x_{n_i},p_{n_i}) + d(p_{n_i},p_{n_j}) + d(p_{n_j},x_{n_j}) \le \frac{1}{4}\epsilon + \frac{1}{4}\epsilon + \frac{1}{4}\epsilon < \epsilon
    .\] 
    and since this holds for any $\epsilon > 0$ and any sequence $(x_n)_n \subset \overline{A}$, By Lemma 1 we have shown that
    every sequence $(x_n)_n \subset \overline{A}$  admits a cauchy subsequence $(x_{n_k})_k$ and further that since  $\overline{A}$ is closed and complete, 
    every cauchy sequence converges and therefore every sequence in $\overline{A}$ admits a convergent subsequence. This implies that $\overline{A}$ is sequentially compact
    which as mentioned previously implies that $\overline{A}$ is compact.
}

\section{Exercise 6}
\emph{
    Construct a compact set of real numbers whose limit points form a countable set
}
\solution{
    Consider the set
    \[
        E = \{\frac{1}{n} + \frac{1}{m}: n,m \in \mathbb{N}\} \cup \{0\} \cup \{\frac{1}{n}: n \in \mathbb{N}\}
    .\] 
    To show this set is compact it suffices to show that it is closed since it is a subset of $[0,1]$ and so is sequentially compact if all limit points lie in the set.\\
    Consider any non constant sequence ommitting all repeating points such $(x_n)_n$ such that $\lim_{n\to \infty}x_n \rightarrow x \in \mathbb{R}$ exists. Every element of the sequence is
    either $(\frac{1}{n_k} + \frac{1}{m_k})$ or $\frac{1}{n}$ or $0$ but $0$ can only happen once so we can ignore it. Similarly if infinitely many terms exist of the form $\frac{1}{n}$ the limit must surely be zero
    since we can always find a larger value of $n$ getting the sequence closer within $\epsilon$ of zero for any epsilon thus we can consider only the sequences with finitely many terms of the form  $\frac{1}{n}$ or $0$.
    Since rearranging a sequence preserves its limit we can rearrange the sequence ignoring all of those terms and in such a way that for any $(\frac{1}{n_k} + \frac{1}{m_k}), (\frac{1}{n_{k+1}} + \frac{1}{m_{k+1}}), n_k + m_k \le n_{k+1} + m_{k+1} $.\\
    This value approaches infinity  for large terms in the sequence and this can happen in one of two ways. Either one term $n_k$ or $m_k$ can go to infinity or both. If both go to infinity the limit is zero. If only one term goes to infinity
    then WLOG $n_k$ goes to infinity and the limit is  $\lim_{n\to \infty} \frac{1}{m_k}$  where this limit exists and is not zero since $m_k$ must be bounded and is increasing. Additionally it must be achieved after finitely many points since $m_k$ can take on values only in $ \mathbb{N}$
}
\begin{comment}
\\solution{
    The cantor set meets this criteria. Note that in $ \mathbb{R}$ for any nested sequence of closed sets $S_1 \supset S_2 \supset S_3 \supset \dots$
    their intersection is nonempty. I define the left set of a finite union of intervals $S = [a_1,b_1] \cup [a_2,b_2] \cup ... \cup [a_n,b_n]$ to be $l_S := \bigcup_{i=1}^{n}[a_i, a_i + \frac{b_i-a_i}{3}]$ and the right set as
    $r_S := \bigcup_{i=1}^{n}[a_i+2\frac{b_i-a_i}{3}, b_i]$. I define the cantor set Recursively. The set starts as the interval $C_1 = [0,1]$ then, $C_2 = l_{C_1} \cup r_{C_1}$,
    $C_n = l_{C_{n-1}} \cup r_{C_{n-1}}$ for all $n \in \mathbb{N}$. And $C = \bigcap_{i=1}^{\infty}C_{i}$. $C$ is closed since it is a countable intersection of closed sets and so C is complete since it is in $ \mathbb{R}$.
    I will show that $C$ is compact using the same argument for intervals in class. Assume for contradiction that $C$ is not compact, then there exists some open cover  $\{G_\alpha\}_{\alpha \in A}$ such that no finite subcover exists.
    Then consider the sets $L_1 = C \cap [0,\frac{1}{3}], R_1 = \cap [\frac{2}{3},1]$. One of these sets must not have a finite cover. Set $D_1$ to be this set and $[A_1,B_1]$ the interval it is defined on. Inductively 
    this holds considering $L_n = D_{n-1} \cap [A_{n-1}, A_{n-1} + \frac{B_{n-1}-A_{n-1}}{3}] , R_n = D_{n-1} \cap [2\frac{B_{n-1}-A_{n-1}}{3}, B_{n-1}]$. At each stage there must be no finite subcover, however
    \[
        \bigcap_{n=1}^{\infty}D_n
    .\] 
    is nonempty, in particular there exists some point $x$ in the intersection. This $x$ is in $G_\alpha$ for some $\alpha$ and since $G_\alpha$ is open there exists some $\epsilon$ such that $N_\epsilon(x) \subset G_\alpha$.
    Since the "width"(b-a) of $D_n$  decreases proportional to $\frac{1}{3}^{n}$ there exists some $n$ such that $D_n \subset N_\epsilon(x)$ and thus $D_n \subset G_\alpha$ which means there
    exists a finite cover of $D_n$ which is a contradiction. Thus there is always a finite cover of $C$ and  $C$ is compact. Additionally all limit points of $C$ are rational. This is because for any irrational number $x \in [0,1]$ such that
     $x$ is a limit point, then for each $C_n$ $x \in C_n$ and $x$ is in some interval not on the endpoints in $C_n$ since all endpoints are rational, the intervals in $C_n$ have width $\frac{1}{3}^{n}$ so there exists an interval in some $C_n$ such that $x_n$
}begin{comment}
\end{comment}
\section{Exercise 7}
\emph{
    Regard $ \mathbb{Q} $ the set of all rational numbers, as a metric space, with $d(p,q) = |p-q|$. Let
    $E$ be the set of all $p \in \mathbb{Q}$ such that $2 < p^2 < 3$. Show that $E$ is closed and bounded in $ \mathbb{Q}$, but that $E$ is
    not compact. Is $E$ open in $ \mathbb{Q}$
}
\solution{
    It is easiest to show that $E$ is bounded, for consider any $p \in E$, $2 < p^2 < 3$ then if $d(p,q) > 10 \iff |p-q| > 10$
    this implies that  $|q| + |p| > |p-q| > 10$ and $2q^2 + 2p^2 > q^2 +2|q||p| + p^2 > 100$ which further implies that $2q^2 + 4 > 100$ and that
    $q^2 > 48$ so $q \not\in E$.\\
    To show that  $E$ is closed we can show that $E^{c}$ is open, the set
    \[
    E^{c} = \{q \in \mathbb{Q}:
        \begin{cases}
            q^2 &\le 2 \\
            q^2 &\ge 3 
    \end{cases} \}
    .\] 
    for any $p \in E^{c}$ there are two cases. If $p^2 \le 2$ then $p^2 < 2$ since $p^2 \ne 2$ since $\sqrt{2}$ is irrational. In particular
    \[
    p \in (-\sqrt{2},\sqrt{2})
    .\] 
    this set is open in $  \mathbb{R}$ and so for this $p$ there exists some $ \epsilon$ such that 
    \[
    N_\epsilon(p) \subset (-\sqrt{2},\sqrt{2}) \subset \mathbb{R}
    .\] 
    but this same epsilon works in $\mathbb{Q}$ for any $q \in \mathbb{Q}$ such that $|q-p| < \epsilon$ implies that $q \in (-\sqrt{2},\sqrt{2}) \subset \mathbb{R}$ since $q \in \mathbb{Q}$
    $q \in (-\sqrt{2},\sqrt{2})$ in $ \mathbb{Q}$.
    Similarly if $p^2 \ge 3$ then $p^2 > 3$ since $p^2 \ne 3$ since $\sqrt{3}$ is irrational. In particular
    \[
    p \in (-\infty,\sqrt{3}) \cup (\sqrt{3}, \infty)
    .\] 
    this set is open in $ \mathbb{R}$ and the argument is identical.
    thus $E^{c}$ is open in $ \mathbb{Q}$ and now we know $ E$ is bounded and closed.
    However $E$ is not compact since for any $\epsilon > 0$ there exists a rational number such that $\sqrt{3} > q \ge \sqrt{3} - \epsilon$ 
    picking a sequence such that
    \[
        (x_n)_n \hspace{1cm} \sqrt{3} > x_n \ge \sqrt{3} - \frac{1}{n}
    .\] 
    This sequence converges to $\sqrt{3}$ in $ \mathbb{R}$ and so the sequence does not converge in  $ \mathbb{Q}$ and there does not exist a subsequence that converges
    since if it did converge it would also converge to $ \sqrt{3}$. Thus $E$ is not sequentially compact and is not compact.
    $E$ is open in $ \mathbb{Q}$ since $E = (-\sqrt{3}, -\sqrt{2}) \cup (\sqrt{2}, \sqrt{3})$ open sets in $Q$ since open intervals in $ \mathbb{Q}$ are open.
}

\end{document}
