\documentclass[11pt]{article}

\newcommand{\cnum}{Math 131BH}
\newcommand{\ced}{Spring 2025}
\newcommand{\ctitle}[4]{\title{\vspace{-0.5in}\cnum, \ced\\Homework Set #1: #2}\author{\vspace{-0.35in}\\#3, #4}}
\usepackage{enumitem}
\usepackage[usenames,dvipsnames,svgnames,table,hyperref]{xcolor}
\usepackage{amsmath, amsfonts}
\usepackage{graphicx} % Required for inserting images
\usepackage{float}
\usepackage{listings}
\usepackage{xcolor}
\usepackage{hyperref}
\usepackage{comment}
\usepackage{mathtools}


\renewcommand*{\theenumi}{\alph{enumi}}
\renewcommand*\labelenumi{(\theenumi)}
\renewcommand*{\theenumii}{\roman{enumii}}
\renewcommand*\labelenumii{\theenumii.}

\definecolor{codegreen}{rgb}{0,0.6,0}
\definecolor{codegray}{rgb}{0.5,0.5,0.5}
\definecolor{codepurple}{rgb}{0.58,0,0.82}
\definecolor{backcolour}{rgb}{0.95,0.95,0.92}
\definecolor{header}{HTML}{205459}
\definecolor{solution}{HTML}{32848c}
\DeclarePairedDelimiter{\norm}{\lVert}{\rVert}


\newcommand{\solution}[1]{{{\textcolor{header}{Solution:} \textcolor{solution}{#1}}}}
\setlist[enumerate]{label=(\alph*)}

\lstdefinestyle{mystyle}{
    backgroundcolor=\color{backcolour},
    commentstyle=\color{codegreen},
    keywordstyle=\color{magenta},
    numberstyle=\tiny\color{codegray},
    stringstyle=\color{codepurple},
    basicstyle=\ttfamily\footnotesize,
    breakatwhitespace=false,
    breaklines=true,
    captionpos=b,
    keepspaces=true,
    numbers=left,
    numbersep=5pt,
    showspaces=false,
    showstringspaces=false,
    showtabs=false,
    tabsize=2
}


\begin{document}
\section{Lemma}
\[
    f: \mathbb{R} \rightarrow \mathbb{R}^2 \hspace{0.5cm} f(x) = (f_1(x),f_2(x)) \hspace{0.5cm} \text{continuous} \implies f_1: \mathbb{R} \rightarrow \mathbb{R} \text{ continuous}
.\] 
Assume $f$ is continuous. For any point $p \in \mathbb{R}$, $\epsilon > 0$ there exists $  \delta > 0$ such that $|p-q| < \delta \implies \norm{f(p) - f(q)} = \norm{(f_1(p)-f_1(q),f_2(p)-f_2(q))} < \epsilon$ but
\[
\sqrt{a^2 + b^2} \ge |a|
.\] 
so this implies that when $|p-q| < \delta$
\[
|f_1(p) - f_1(q)| < \epsilon
.\] 
thus $f_1$ is continuous.


\section{Theorem}
\[
    U \subset \mathbb{R} \;\; \text{closed}
.\] 
\[
    E = \{(x,g(x)): x \in U\} \subset \mathbb{R}^2
.\] 
\[
    g: U \rightarrow \mathbb{R} \hspace{1cm}  \text{not continuous} \implies E \text{ not path connected}
.\] 
$g$ not continuous implies that there exists some point $p \in U$ and some $\epsilon > 0$ such that for every $\delta > 0$ there exists $q, |p-q| < \delta, \norm{g(p)-g(q)} > \epsilon$ 
Assume for contradiction that $E$ is path connected, then for any $x \in U$ there exists continuous  $f: [0,1] \rightarrow E$ such that $f(0) = (x,g(x)), f(1) = (p,g(p))$. Consider
$U \cap (p, \infty)$ and $U \cap (-\infty, p)$ one of these sets must be nonempty and contain infinitely many points like $q$ as described earlier. Assume without loss of generality that the set
$U \cap (p, \infty)$ contains infinitely such $q$ and pick one of them to be $x$.\\

if $a < b$ and $f(a) = (A,g(A)), f(b) = (B,g(B))$  then for any $C$ between $A$ and $B$, there exists  $a < c < b$ such that $f(c) = (C, g(C))$ this is because $f(t) = (f_1(t),f_2(t))$ and $f_1(t)$ is continous by Lemma 1 so  $f_1((a,b))$ is connected and $A,B \in f_1((a,b))$ so $U \subset f_1((a,b))$ where $U$ is the open interval between $A$ and $B$ where in particular $C \in U \implies C \in f_1((a,b)) \implies (C,g(C)) \in f((a,b))$.\\

Let $t$ be $\inf_{t: f(t) = (p,g(p))}(t)$, the minimum $t$ that satisfies $f(t) = (p,g(p))$. Importantly if $\tau < t \implies f(\tau) = (a_1,g(a_1)), a_1 > p$ By the previous paragraph\\

Since $f$ is continuous there exists some $\delta_f > 0$ such that $|y-t| < \delta_f \implies \norm{(p,g(p)) - f(y)} < \epsilon$ pick  $\tau = t-\frac{\delta_f}{2}$ then $f(\tau) = (a_1,g(a_1)), a_1 > p$. Pick
$q$ such that $q > p$, $|p-q| < |a_1-p|$, and $\norm{g(p) - g(q)} > \epsilon$ then by Lemma there exists some $\eta$ such that $\tau \le \eta < t$ such that  $f(\eta) = (q,g(q))$ but
\[
    \eta \in N_{\delta_f}(t) \hspace{1cm} \norm{f(\eta) - (p,g(p))} = \norm{(q,g(q)) - (p,g(p))} \ge |g(p)-g(q)| > \epsilon
.\] 
This is a contradiction thus $E$ is not path connected.

\section{$\sin(\frac{1}{x})$}
\[
f(x) = 
\begin{cases}
    \sin(\frac{1}{x}) & x > 0\\
    0 & x \le 0
\end{cases}
.\] 
is discontinuous at $x=0$ therefore the graph $E := \{(x,f(x)): x \in \mathbb{R}\}$ is not path connected.


\end{document}
