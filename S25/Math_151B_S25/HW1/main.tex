\documentclass[a4paper,12pt]{scrartcl} % From KOMA-script
\usepackage[margin=1in]{geometry}
\usepackage{xcolor}
\usepackage{graphicx}
\usepackage{fancyhdr}
\usepackage{lipsum} % For filler text
\usepackage{amsthm}
\usepackage{times}
\usepackage{microtype}
\usepackage{amsmath, amssymb, amsfonts, amssymb, float, enumitem}
\definecolor{darkbg}{HTML}{1E1E1E}
\definecolor{darktext}{HTML}{E0E0E0}
\definecolor{theoremcolor}{HTML}{d3dce5}
\definecolor{answercolor}{HTML}{e9dfc0}
\definecolor{sectioncolor}{HTML}{dec3c3}
\pagecolor{darkbg}
\color{darktext}
\newenvironment{solution}
  {\par\color{answercolor}\textbf{Solution:}\ }
  {\par}
\newenvironment{prf}
  {\par\textbf{Proof:}\ }
  {\par}

\newcounter{customcounter}
\newcommand{\setcustomcounter}[1]{\setcounter{customcounter}{#1}}

\newtheoremstyle{darktheorem}
  {\topsep}{\topsep}
  {\itshape\color{theoremcolor}}{}
  {\bfseries\color{theoremcolor}}{.}{.5em}{}
\theoremstyle{darktheorem}
\newtheorem{theorem}{Theorem}[section]
\newtheorem{lemma}[theorem]{Lemma}
\newtheorem{proposition}[theorem]{Proposition}
\newtheorem{corollary}[theorem]{Corollary}
\newtheorem{definition}[theorem]{Definition}
\newtheorem{example}[theorem]{Example}
\newtheorem{exercise}[customcounter]{Exercise}

\usepackage{fancyhdr}
\pagestyle{fancy}
\setlist[enumerate]{label=(\alph*)}

% Clear default headers
\fancyhf{}
\renewcommand{\headrulewidth}{0pt}

% Set Author and Date in the top left
\fancyhead[L]{\textcolor{darktext}{\small Asher Christian \\ 006-150-286 \\ \today }}

\begin{document}


\title{\color{sectioncolor}HW 1 Math 151B}
\author{}
\date{}
\maketitle

% Apply fancyhdr on the title page
\thispagestyle{fancy}

\begin{exercise}
    Let $\lambda = a + ib$ where $a,b \in \mathbb{R}$ and $i = \sqrt{-1}$. The solution to
    IVP:
     \[
         y' = \lambda y \hspace{2cm} y(0) = 1
    .\] 
    is given by $y(t) = e^{\lambda t}$ show that the modulus of the solution is $|y(t)| = e^{at}$. Then, deduce
    that
    \[
    |y(t)|
    \begin{cases}
        \rightarrow 0 & a < 0\\
        = 1 & a = 0  \hspace{2cm} t \rightarrow \infty\\
        \rightarrow \infty & a > 0
    \end{cases}
    .\] 
    the modulus of a complex number $z = \alpha + i \beta$ is $|z| = \sqrt{\alpha^2 + \beta^2}$ and $e^{i\theta} = \cos(\theta) + i\sin(\theta)$
\end{exercise}
\begin{solution}
    \[
    y(t) = e^{\lambda t} = e^{(a + bi)t} = e^{at}(e^{bit}) = e^{at}(\cos(bt) + i\sin(bt))
    .\] 
    \[
    |y(t)| = e^{at}\sqrt{\cos^2(bt) + \sin^2(bt)} = e^{at}
    .\] 
    if $a < 0$ $at$ is monotone decreasing with no lower bound and $e^{-\infty}$ approaches zero so  
    \[
    \lim_{t\to \infty} e^{at} = 0
    .\] 
    if $a = 0$ then $e^{at} = e^{0} = 1$ for all $t$ and
    \[
    \lim_{t\to \infty}1 = 1
    .\] 
    if $a > 0$ then $at$ grows without bound and approaches $\infty$ with $e^{n}$ approaching infinity as well
    so 
    \[
    \lim_{t\to \infty}e^{at} = \infty
    .\] 
\end{solution}

\begin{exercise}
    \leavevmode
    \begin{enumerate}[label = (\alph*)]
        \item Let $D = \{(t,y) : a \le t \le b,c \le y \le d\}$. Suppose $f \in C^{1}(D)$.
            Show that $f$ is Lipschitz continuous in $D$.
        \item Show that $f(t,y) = \frac{t^2y^2}{1+t^2}$ is Lipschitz continuous in $D = \{(t,y): t_0 \le t \le t_f, -\delta \le y \le \delta\}$
    \end{enumerate}
\end{exercise}
\begin{solution}
    \begin{enumerate}[label = (\alph*)]
        \item  $f \in C^{1}(D)$ implies that
            \[
            \frac{df}{dt} = f_t
        .\] exists and is continuous on $D$.
        since $D$ is a compact set, and $f_t$ is continuous
        \[
            \sup_{p \in D}|f_t(p)| = M
        .\] 
        exists.\\
        consider the following for any $y_1 < y_2 \in [c,d]$
            \[
                \frac{f(t,y_1) - f(t,y_2)}{y_1-y_2} = f_t(t,\xi)
            .\] 
            for some $\xi \in (y_1,y_2)$ and in particular
            \[
            |f(t,y_1) - f(t,y_2)| = |y_1-y_2||f(t,\xi)| \le M|y_1-y_2|
            .\] 
            and so $f$ is Lipschitz continuous on $D$
        \item 
            \[
            |\frac{df}{dt}| = 2y^2|\frac{t}{(1+t^2)^2}| \le 2y^2 \le 2\delta^2
            .\] 
            this is because $|\frac{t}{(1+t^2)^2}| \le 1$ which can be verified by considering the cases where $|t| < 1$
            and $|t| \ge 1$ and since  $|y| \le \delta$ by nature of $D$, and so by the nature
            of the previous part of the question, $f$ is lipschitz continuous.
    \end{enumerate}
\end{solution}

\begin{exercise}
    \begin{enumerate}
        \item Apply Forward Euler method to the IVP $y'(t) = \lambda y(t)$ with $\lambda = -10$ and 
            $y(0) = 1$. Show that the numerical solution is sgiven by
            \[
                y_n = (1-10h)^{n}, \hspace{1cm} n \ge 0
            .\] 
        \item Compute $y_1,y_2,y_3$ for $h = \frac{1}{6}$ and $h=\frac{1}{12}$ respectively.
            Comment on your results.
        \item What is the largest value of $h$ that can be used for $\lambda = -10$ such that the numerical solution
            $y_n \ge 0$ for all $n \ge 1$?
    \end{enumerate}
\end{exercise}
\begin{solution}
    \begin{enumerate}
        \item we have $y(t_0+h) = y(t_0) + hy'(t_0) + O(h^2) = y(t_0) -10hy(t_0) + O(h^2) = y(t_0)(1-10h) $
        \item 
            with $h = \frac{1}{6}$,
            \begin{align*}
                y_1 = y(\frac{1}{6}) &= 1(1-\frac{5}{3}) = -\frac{2}{3}\\
                y_2 = y(\frac{2}{6}) &= -\frac{2}{3}(1-\frac{5}{3}) = \frac{4}{9}\\
                y_3 = y(\frac{3}{6}) &= \frac{4}{9}(-\frac{2}{3}) = -\frac{8}{27} 
            \end{align*}
            with $h = \frac{1}{12}$ 
            \begin{align*}
                y_1 = y(\frac{1}{12}) &= 1(1-\frac{10}{12}) = \frac{1}{6}\\
                y_2 = y(\frac{1}{6}) &= \frac{1}{6}(\frac{1}{6}) = \frac{1}{36}\\
                y_3 = y(\frac{1}{4}) &= \frac{1}{36}(\frac{1}{6}) = 216
            \end{align*}
        \item since $y_0$ is positive for $y_n$ to change signs $1-10h$ must be negative. This means that $h > \frac{1}{10}$ so $h = \frac{1}{10}$ 
            is the largest $h$ value that does not change signs. 
    \end{enumerate}
\end{solution}
\begin{exercise}
    Consider the IVP
    \[
        y''(t) + y(t)y'(t) + 4y(t) = t^2, \hspace{1cm} t > 0
    .\] 
    \[
        y(0) = 0, \hspace{1cm} y'(0) = 1
    .\] 
    \begin{enumerate}
        \item Rewrite the IVP as a first order system of ODE.
        \item Derive Forward Euler method for the system that you have derived in $(a)$
    \end{enumerate}
\end{exercise}
\begin{solution}
    \begin{enumerate}
        \item First let $v =y$ and $u = y'$ then
            \[
            u' = y'' = t^2 - vu -4v
            .\] 
            \[
            v' = y' = u 
            .\] 
            $v(0) = 0, u(0) = 1$
            in matrix form we have
             \[
            \begin{pmatrix} v'(t) \\ u'(t) \end{pmatrix} = \begin{pmatrix} u \\ t^2 - vu - 4v  \end{pmatrix} \\
            .\] 
        \item For forward Euler we have
            \begin{align*}
                y_{n+1} &= y_{n} + hf(y_{n})\\
                \begin{pmatrix} v_{n+1} \\ u_{n+1} \end{pmatrix} &= \begin{pmatrix} v_n \\ u_n \end{pmatrix} + h \begin{pmatrix} u_n \\ t_n^2 - v_nu_n -4v_n \end{pmatrix} 
            \end{align*}
            solving with $y_1,y_2$ to approximate $y(0.1), y(0.2)$ we get
            \begin{align*}
                \begin{pmatrix} v_1 \\ u_1 \end{pmatrix} &= \begin{pmatrix} 0 \\ 1 \end{pmatrix} + 0.1 \begin{pmatrix} 1 \\ 0^2 - 0(1) - 4(0) \end{pmatrix} \\
                &= \begin{pmatrix} 0.1 \\ 1 \end{pmatrix} \\
                \begin{pmatrix} v_2 \\ u_2 \end{pmatrix} &= \begin{pmatrix} 0.1 \\ 1 \end{pmatrix}  + 0.1 \begin{pmatrix} 1 \\ 0.1^2 - 0.1(1) - 4(0.1) \end{pmatrix} \\
                                                         &= \begin{pmatrix} 0.2 \\ 0.951 \end{pmatrix} 
            \end{align*}
            so we get $y(0.2) \approx y_2 = 0.2$
    \end{enumerate}
\end{solution}
\begin{exercise}
    Find the local trunctation error of the Backward Euler method. Show that it is first order accurate
\end{exercise}
\begin{solution}
    Recall backward Euler is given by
    \[
        y_{n+1} = y_n + hf(t_{n+1},y_{n+1})
    .\] 
    and recall the taylor expansion about $x=t_{n}$ with remainder, 
     \[
         y(t_{n}) = y(t_{n+1}) - hy'(t_{n+1}) + \frac{h^2}{2}y''(\xi) 
    .\] 
    \[
        y(t_{n+1}) = y(t_n) + hy'(t_{n+1}) - \frac{h^2}{2}y''(\xi)
    .\] 
    for some $\xi \in [t_n,t_n +h] = [t_n, t_{n+1}]$. Now assuming that $y_n = y(t_n)$ we get
     \[
         \tau_{n+1} = y(t_{n+1}) - y_{n+1} = y(t_n) + hy'(t_{n+1}) - \frac{h^2}{2}y''(\xi) - y(t_n) - hf(t_{n+1},y_{n+1})
    .\] 
    \[
        \tau_{n+1} = h(f(t_{n+1},y(t_{n+1}))-f(t_{n+1},y_{n+1})) - \frac{h^2}{2}y''(\xi)
    .\] 
    Now assuming further that $f$ is Lipschitz continuous in  $y$ with Lipschitz constant  $K$ we get
    \[
        \tau_{n+1} = hK(y(t_{n+1}) - y_{n+1}) - \frac{h^2}{2}y''(\xi)
    .\] 
    \[
        \tau_{n+1} = hK(\tau_{n+1}) - \frac{h^2}{2}y''(\xi)
    .\] 
    \[
        \tau_{n+1} = -\frac{h^2y''(\xi)}{2(1-hK)} = O(h^2)
    .\] 
    thus $BE$ is first order accurate.
\end{solution}


\end{document}
