\documentclass{article}
\usepackage{amsmath} % For math equations
\usepackage{amsfonts} % For math fonts
\usepackage{amssymb} % For math symbols
\usepackage{float}
\usepackage{enumitem}
\usepackage{graphicx}
\setlist[enumerate,1]{label=\arabic*.}
\setlist[enumerate,2]{label=\alph*.,itemindent=2em}
\DeclareMathOperator{\Ima}{Im}
\DeclareMathOperator{\tr}{Tr}

\title{HW 7 - 115B}
\author{Asher Christian 006-150-286}
\date{ 03.03.25}

\begin{document}
    \maketitle
    \section{Exercise 1}
    \emph{
        Prove all orthogonal projections are self adjoint
    }\\
    Let $T: V \rightarrow V$ be an orthogonal projection on $V$ a finite dimensional 
    inner product space onto $W \subset V$ subspace along $W^{\perp}$ pick an orthogonal basis
    $B_1 = \{v_1,...,v_n\}$ for $W$ and an orthonormal basis $B_2 = \{u_1,..,u_m\}$ of $W^{\perp}$
    then $B =\{v_1,..,v_n,u_1,..,u_n\}$ is an orthonormal basis for $V$. This is because each $u_i$ is
    orthonogonal to all $v_i$ and thus linearly independent and since $V = W \oplus W^{\perp}$ 
    the set is a basis. in particular
    \[
        [T]_B = \begin{pmatrix} 
            1 & 0 & ... & 0 & 0\\
            0 & 1 & ... & 0 & 0\\
            \vdots & \vdots & \ddots & \vdots & \vdots\\
            0 & 0 & ... & 0 & 0\\
            0 & 0 & ... & 0 & 0
        \end{pmatrix} 
    .\] 
    and since $B$ is orthonormal, $[T^{*}]_B =  [T]_B^{*} = [T]_B$ since the matrix is diagonal with all real entries so $T = T^{*}$ is self adjoint

    \section{Exercise 2}
    \emph{
        Let $T$ be an orthogonal(unitary) operator on a finite-dimensional real(complex) inner product
        space $V$. If $W$ is a $T$-invariant subspace of $V$. prove the following
        \begin{enumerate}[label = (\alph*)]
            \item $T|_W$ is an orthogonal(unitary) operator on $W$.\\
            \item $W^{\perp}$ is a $T$-invariant subspace of $V$.\\
            \item $T|_{W^{\perp}}$ is an orthogonal(unitary) operator
        \end{enumerate}
    }
    \begin{enumerate}[label = (\alph*)]
        \item 
            for any $w \in W$, $||T|_Ww|| = ||Tw|| = ||w||$ so  $T|_W$ preserves norms and is unitary
        \item 
            by exercise 7 HW 5 part (b)  $W^{\perp}$ is $T^{*}$ invariant since orthogonal/unitary operators are normal.
            And by exercise 8 HW 5 (I proved it for general normal not necessarily complex), since $W^{\perp}$ is $T^{*}$ invariant it is also $(T^{*})^{*} = T$ invariant and thus $W^{\perp}$ is $T$ 
            invariant.
        \item 
            for any $w \in W^{\perp}$, $||T|_{W^{\perp}}w|| = ||Tw|| = ||w||$ so $T|_{W^{\perp}}$ preserves
            norms and is orthogonal(unitary)
    \end{enumerate}
    \section{Exercise 3}
    \emph{
        Let $V$ be a real inner product space of dimension two. Prove that rotations, reflections
        and compositions of rotations and reflections are orthogonal operators.
    }
    Note that composition orthogonal operations is orthogonal for if $U,T : V \rightarrow V$ both orthogonal
    \[
    ||UTv|| = ||Tv|| = ||v||
    .\] 
    Inductively this holds for any number of compositions.
    preserves norms is orthogonal. And since in class we classified all orthogonal operators on $ \mathbb{R}^2$ to be either rotations or reflections.
    It is clear that since the composition of any rotation and/or reflection is orthogonal and thus a rotation or reflection.

    \section{Exercise 4}
    \emph{
        For any real number $\theta \in \mathbb{R}$ let $A_{\theta} = \begin{pmatrix} \cos(\theta) & \sin(\theta) \\ \sin(\theta) & -\cos(\theta) \end{pmatrix} $.
        \begin{enumerate}[label = (\alph*)]
            \item Prove that $L_{A_{\theta}}$ is a reflection.
            \item Find the subspace of $ \mathbb{R}^2$ about which $L_{A_{\theta}}$ reflects.
        \end{enumerate}
    }
    \begin{enumerate}[label = (\alph*)]
        \item 
        Let
        \[
            r_1 = \begin{pmatrix} \cos(\frac{\theta}{2}) \\ \sin(\frac{\theta}{2}) \end{pmatrix} r_2 = \begin{pmatrix} -\sin(\frac{\theta}{2}) \\ \cos(\frac{\theta}{2})\end{pmatrix} 
        .\] 
        then $r_1$ and $r_2$ form an orthonormal basis for $ \mathbb{R}^2$ and note that
        \[
            R = \begin{pmatrix} 1 & 0 \\ 0 & -1 \end{pmatrix} 
        .\] 
        is clearly a reflection about the line spanned by first basis vector
        Then using change of basis we have
        \[
        \begin{pmatrix} 
            \cos(\frac{\theta}{2}) & -\sin(\frac{\theta}{2}) \\
            \sin(\frac{\theta}{2}) & \cos(\frac{\theta}{2})
        \end{pmatrix} 
        \begin{pmatrix} 1 & 0\\ 0 & -1 \end{pmatrix} 
        \begin{pmatrix} 
            \cos(\frac{\theta}{2}) & -\sin(\frac{\theta}{2}) \\
            \sin(\frac{\theta}{2}) & \cos(\frac{\theta}{2})
        \end{pmatrix}^{-1}
        .\] 
        is a reflection represented in the standard basis about the line spanned by $r_1$.
        First note that since the last matrix is a rotation, and orthogonal its inverse is its transpose
        and by the multiplication we get
        \[
            \begin{pmatrix} 
                \cos(\frac{\theta}{2}) & \sin(\frac{\theta}{2})\\
                \sin(\frac{\theta}{2}) & -\cos(\frac{\theta}{2})
            \end{pmatrix}
            \begin{pmatrix} 
                \cos(\frac{\theta}{2}) & \sin(\frac{\theta}{2}) \\
                -\sin(\frac{\theta}{2}) & \cos(\frac{\theta}{2})
            \end{pmatrix}
            =
            \begin{pmatrix} 
                \cos^2(\frac{\theta}{2}) - \sin^2(\frac{\theta}{2}) & 2\cos(\frac{\theta}{2})\sin(\frac{\theta}{2})\\
                2\cos(\frac{\theta}{2})\sin(\frac{\theta}{2}) & \sin^2(\frac{\theta}{2}) - \cos^2(\frac{\theta}{2})
            \end{pmatrix} 
        .\] 
        and by double angle formula this is equivalent to
        \[
        \begin{pmatrix} 
            \cos(\theta) & \sin(\theta) \\
            \sin(\theta) & -\cos(\theta)
        \end{pmatrix} 
        .\] 
    \item in fact in the previous problem we shows that the subspace of $ \mathbb{R}^2$ about which $L_{A_{\theta}}$ reflects
        is the first basis vector which is precisely
        \[
        r_1 = \begin{pmatrix} \cos(\frac{\theta}{2}) \\ \sin(\frac{\theta}{2}) \end{pmatrix} 
        .\] 
    \end{enumerate}
    \section{Exercise 5}
    \emph{
        For any real number $\theta \in \mathbb{R}$ define $R_{\theta}: \mathbb{R}^2 \rightarrow \mathbb{R}^2$ to
        be the linear transformation given by left multiplication by the matrix
         \[
        \begin{pmatrix} 
            \cos(\theta) & -\sin(\theta) \\
            \sin(\theta) & \cos(\theta)
        \end{pmatrix} 
        .\] 
        \begin{enumerate}[label= (\alph*)]
            \item Prove that any rotation on $ \mathbb{R}^2$ is of the form $R_{\theta}$ for some
                $\theta \in \mathbb{R}$
            \item Prove that $R_{\theta}R_{\theta'} = R_{\theta + \theta'}$ for any $\theta, \theta' \in \mathbb{R}$ 
            \item show that any two rotations on $ \mathbb{R}^2$ commute
        \end{enumerate}
    }
    \begin{enumerate}[label = (\alph*)]
        \item Consider any rotation $T$ on $ \mathbb{R}^2$ let
            \[
            r_1 = \begin{pmatrix} \cos(\theta) \\ \sin(\theta) \end{pmatrix}  = Te_1
            .\] 
            then using trigonometry and the fact that rotations preserve angles
            \[
            r_2 = \begin{pmatrix} -\sin(\theta) \\ \cos(\theta) \end{pmatrix}  = Te_2
            .\] 
            is the only possible value
        \item 
            using euler's identity
            \[
            e^{i\theta} = \cos(\theta) + i\sin(\theta)
            .\] 
            letting $i$ be the second standard basis we see that multiplication by $e^{i\theta}$ is equivalent
            to multiplication by the rotation matrix for any $\begin{pmatrix} a \\ b \end{pmatrix} $.
            \[
                R_\theta \begin{pmatrix} a \\ b \end{pmatrix}  = \begin{pmatrix} a\cos(\theta) - b\sin(\theta) \\ a\sin(\theta) + b\cos(\theta) \end{pmatrix} 
            .\] 
            and
            \[
                (a+bi)e^{i\theta} = a\cos(\theta) - b\sin(\theta) + (a\sin(\theta) + b\cos(\theta))i
            .\] 
            is equivalent
            . In particular for any $\begin{pmatrix} a \\ b \end{pmatrix} \in \mathbb{R}^2$ the corresponding $a +bi$
            \[
                ((a+bi)e^{i\theta})e^{i\theta'} = (a+bi)e^{i(\theta+\theta')}
            .\] 
            is equivalent to rotation by $\theta + \theta'$ degrees.
        \item Using the same notion it is clear that rotations commute. Additionally by the previous argument
            $R_{\theta}R_{\theta'} = R_{\theta+\theta'} = R_{\theta'}R_{\theta}$ since addition is commutative
    \end{enumerate}
    \section{Exercise 6}
    \emph{
        Prove that no orthogonal operator on a two dimensional real inner product space
        can be both a rotation and a reflection
    }\\
    note that for any rotation
    \[
        \det(\begin{pmatrix} \cos(\theta) & -\sin(\theta) \\
        \sin(\theta) & \cos(\theta)
        \end{pmatrix} ) = \cos^2(\theta) + \sin^2(\theta) = 1
    .\] 
    whereas any reflection
    \[
    \det( \begin{pmatrix} 
        \cos(\theta) & \sin(\theta)\\
        \sin(\theta) & -\cos(\theta)
    \end{pmatrix} ) = -\cos^2(\theta) - \sin^2(\theta) = -1
    .\] 
    since the determinant of a trnasformation is independent of the basis, it is impossible for a transformation to have
    determinant of $-1$ and $1$ simultaneously and so it is impossible for a transformation to be both a rotation and a reflection.

    \section{Exercise 7}
    \emph{
        Let $V$ be a finite dimensional real inner product space. Define $T : V \rightarrow V$ via the
        formula $T(\vec(v)) = -\vec(v)$. prove that $T$ is a direct sum of rotations iff the dimension of $V$ is even.
    }
    Pick an orthonormal basis $B = \{v_1,...,v_n\}$ for $V$ then
    \[
        [T]_B = \begin{pmatrix} 
            -1 & 0 & ... & 0\\
            0 & -1 & ... & 0\\
            \vdots & \vdots & \ddots & \vdots\\
            0 & 0 & ... & -1
        \end{pmatrix} 
    .\] 
    in particular for any two $v_i, v_j$ the space $\text{span}\{v_i,v_j\}$ is $T$ invariant and if the space is $W_{ij}$ then
    $T|_{W_{ij}}$ is a rotation namely the matrix 
    \[
        \begin{pmatrix} 
            -1 & 0\\
            0 & -1
        \end{pmatrix} =
        \begin{pmatrix} 
            \cos(\pi) & -\sin(\pi)\\
            \sin(\pi) & \cos(\pi)
        \end{pmatrix} 
    .\] 
    additionally if $i,j,k,l$ are all pairwise distinct then $W_{ij} \cap W_{kl} = \{0\}$ by the basis definition.
    THus if $\dim(V)$ even then we can pair each basis to a unique other basis for  $\frac{n}{2}$ subspaces each with a rotation. However if $\dim(V)$
    is odd then at least one subspace must be one dimensional and in particular since the subspace is spanned by some $v \in V$ with $T(v) = -v$, by definition this transformation
    is a reflection on the one dimensional subspace so $T$ is not a direct sum of rotations.

    \section{Exercise 8}
    \emph{
        Let $V$ be a real inner product space of dimension 2. For any $v,w \in V$ such that 
        $||v|| = ||w|| = 1$, show that there exists a unique rotation $R$ on $V$ such that $R(v) = w$
    }
    Pick an orthonormal basis for $V$ $B = \{v,v_2\}$  with $v$ as above then the inner product on $V$ is equivalent to the 
    inner product on $ \mathbb{R}^2$ using the standard basis as above. In particular every element in $V$ that is
    length 1 is isomorphic to a vector in $ \mathbb{R}^2$ with length 1. We know  that $e_1$ in $ \mathbb{R}^2$ represents $v \in V$.
    since $||w|| = 1$ it must be $ \begin{pmatrix} \cos(\theta) \\ \sin(\theta) \end{pmatrix} $ for some $ \theta \in [0,2\pi)$
    Consider the matrix
    \[
        [T]_B = \begin{pmatrix} \cos(\theta) & -\sin(\theta) \\ \sin(\theta) & \cos(\theta) \end{pmatrix} 
    .\] 
    This transformation sends $[v]_B = e_1$ to $[w]_B$ so  $T$ represented by this matrix sends $v$ to $w$ and 
    this transformation is a reflection.


\end{document}
