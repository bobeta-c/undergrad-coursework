\documentclass{amsart}
\usepackage{amsfonts} % For math fonts
\usepackage{amsmath, amssymb, amsthm}
\usepackage{float}
\usepackage{enumitem}
\usepackage{graphicx}
\setlist[enumerate,1]{label=\arabic*.}
\setlist[enumerate,2]{label=\alph*.,itemindent=2em}
\setlist{topsep=0pt, leftmargin=*, labelsep=1em}

\title{HW 3 - 131AH}
\author{Asher Christian 006-150-286}
\date{11.11.24}

\begin{document}
    \maketitle
    \section*{Exercise 3.1}
    This exact question was asked on the previous homework but I will show a similar solution.
    \\
    $E$ is compact so there is a finite subcover $E = \bigcup_{i=1}^{n}U_{i_j}$. Assume for contradiction that for every $\epsilon > 0$ there is a bad ball
    at center $x_\epsilon \in E$ that is not fully contained in any of the finite subcover. So for any $n \in \mathbb{N} $ we can find $x_n \in E$ with $B_{\frac{1}{n}}(x_n)$ has a
    nonzero intersection with every subcover. Since compactness implies sequential compactness by 3.5 there is a convergent subsequence for this sequence that converges to some $x$ in $E$.
    Take a similar sequence for each $U_{i_j}$ that is in the intersection of the bad balls with the compliment of  $U_{i_j}$. This sequence must also converge to  $a$ because it is in every ball.
    But since each compliment is closed that implies that $x$ is in the compliment of each element of the finite open subcover which implies that it is an element of the empty set which is a contradiction.
    
    


    \section*{Exercise 3.2}
        \emph{We call a metric space sequentially compact if every sequence has a
    convergent subsequence. Prove that a metric space $E$ is sequentially compact if and only if
    every infinite subset has a cluster point.}\\
        For this proof $(i)$ represents the statement $E$ is sequentially compact and $(ii)$ represents the statement
        every infinite subset of  $E$ has a cluster point.
        \\
        First to prove $(i) \Rightarrow (ii)$ consider any infinite subset of $E$, $S := \{s_1,s_2,...\}$
        and the sequence $(x_n)_n$ generated by taking distinct elements $x_n \in S$.
        By the sequential compactness of $E$ we can find a subsequence $(x_{n_k})_k$ converging in $E$ to $x$ with each $(x_{n_k}) \in S$.
        By the definition of convergent sequence for any $\epsilon > 0$ we can find $N \in \mathbb{N} $ such that $d(x,x_{n_k}) < \epsilon \forall n_k > N$ and since the sequence is
        infinite, $x$ is a cluster point of $S$.\\
        To prove $(ii) \rightarrow (i)$ consider any infinite sequence $(x_n)_n$ in $E$ the set $S := \{x_n, n\in \mathbb{N} \}$ is either infinite or it contains one point infinitely many times and a subsequence can be created with only that point that converges. If instead it is infinite it contains a cluster point that we call $x$.
        We can then construct a strictly increasing sequence $(n_k)_k$ such that $d(x_{n_k}, x) < \frac{1}{k}$ This is possible because there are infinitely many points in the sequence that are arbitrarily close to $x$.
    Then the subsequence $(x_{n_k})_k$ converges to  $x$ because for all $\epsilon > 0$ pick $K > \frac{1}{\epsilon }, K \in \mathbb{N} $ and so $d(x_{n_k}, x) < \frac{1}{K} < \epsilon $ for all $k > K$.

    \section{Exercise 3.3}
    This question is solved in 3.1 given the information in 3.5 that proves that sequential compactness is equivalent to compactness in a metric space.

    \section{Exercise 3.4}
        \emph{We call a metric space totally bounded if, for every $\epsilon > 0 $, the metric
    space is the union of a finite numbers of closed balls of radius $\epsilon $. Prove that a metric space
    totally bounded if and only if every sequence has a Cauchy subsequence.}\\
    For this proof $(i)$ will mean Totally bounded and $(ii)$ will mean every sequence has a Cauchy subsequence.\\
    First to prove $(i) \rightarrow (ii)$. Consider any sequence $(x_n)_n$ in a totally bounded metric space $E$ with no element repeating infinitely many times and any $\epsilon > 0$.
    Consider the finite set $S$ of closed balls of radius $\epsilon $ that totally span $E$. There must be at least one ball lets call $B$ with an infinite amount of $x_n$ for if this
    were not the case and each ball had only a finite amount of elements the total sequence would have a finite amount of elements which would be a contradiction. We can consider the subsequence generated by only admitting
    elements if they are in $B$. This sequence satisfies $d(x_n,x_m) < \epsilon $ and thus is cauchy. If instead $(x_n)_n$ has an element that repeats an infinite amount of time a subsequence can be generated with just that element that is Cauchy.
    \\
    To prove $(ii)\rightarrow (i)$  First assume for contradiction that $E$ is not totally bounded. Then there exists an $\epsilon > 0$ such that $E$ is not the union of a finite number of closed balls of radius $\epsilon $.
    Start with any point $p_0$ then there must exist a point $p_1$ such that $d(p_0,p_1) > \epsilon $ for if that were not the case $E$ would be contained in a single ball of radius $\epsilon $ about the point $p_0$.
    Continue iteratively picking $p_2,p_3,...,$ in $E$ such that each $p_n$ is at least $\epsilon $ away from the previous $p_0,p_1,...,p_{n-1}$. This can continue indefinitely because if it stopped there would be a finite amount of balls centered at each $p_n$ with radius $\epsilon $ spanning the entire set.
    Now consider the sequence $(p_n)_n$ this sequence must have no Cauchy subsequence because for each $p_i,p_j, i\ne j, d(p_i, p_j) > \epsilon $. Thus we have arrived at a contradiction which proves that if every sequence has a cauchy subsequence then $E$ must be totally bounded.

    \section{Exercise 3.5}
    \emph{Let E be a metric space. Consider the following relations:
    (i) E is compact. (ii) E is sequentially compact. (iii) E is totally bounded and complete.}
        
    \begin{enumerate}
        \item $(ii) \Leftrightarrow (iii)$\\
            To show $(ii) \rightarrow (iii)$ we first note that all convergent sequences are cauchy so if $E$ is sequentially compact every sequence has a cauchy subsequence and we above proved that if every sequence has a cauchy subsequence that the 
            space is totally bounded. Additionally if every sequence has a convergent subsequence then every cauchy sequence is convergent because all subsequences of a cauchy sequence have the same convergence if it exists and the space must be complete.
            The opposite direction is also immediate from the above proof for if the space is totally bounded then every sequence has a cauchy subsequence and if $E$ is complete then
            every cauchy sequence converges so putting them together every sequence has a convergent subsequence.
        \item $(ii) \Leftrightarrow (i)$.\\
            We have already shown that $(ii) \Leftrightarrow (iv)$ with $(iv)$ being every infinite subset has a cluster point. It suffices
            then to show that $(i) \rightarrow (iv)$ and $(ii) \rightarrow (i)$. First to show $(i) \rightarrow (iv)$ assume for contradiction that there is an infinite subset $S$ with no cluster point.
            Then for any point $p$ in $E$ we can take an open ball at its center that intersects $S$ only finitely many points.
            Since $E$ is the union of all balls of this type $E$ is a union of an infinite amount of open balls but since $E$ is compact it must be the union of a finite number of these
            balls and so  $S$ must be finite which is a contradiction.
            \\
            We can then show that $(ii) \rightarrow (i)$ to complete the proof. Assume for the sake
            of contradiction that $E$ is sequentially compact but not compact. So there exists some infinite covering  $\{U_i\}_{i\in I}$ with no finite subcover covering  $E$.
            This means that for any $\bigcup_{i \in I} U_i$ does not cover  $E$. Let $I_j$ be an subset of the ordered set of $I$ with $j$ elements such that $I_j \subset I_k$ with $j < k$.
            take a sequence $(x_n)_n$ with each $x_n \not\in \bigcup_{i \in I_n} U_i$. Since no finite subcover exists this process can continue indefinitely with each $x_n$ not being in the finite subcover of $n$ $U_i's$.
            By sequential compactness there must be a subsequence  $(x_{n_k})_k$ that converges to $x$ in $E$. And thus $x \in U_p$ for some $p \in I$. and because each set $U_i$ is open and the subsequence converges there must
            be an integer $N$ s.t. for all $k \ge N$ $x_{n_k} \in U_p$. This is a contradiction because each $x_n$ lies outside of a finite collection of $U_i$ which means that our assumption that $E$ is not compact is false therefore
            showing the $(ii) \rightarrow (i)$

    \end{enumerate}
    \section{Exercise 3.6}
    \emph{Discuss the continuinty of the function $f : \mathbb{R} \rightarrow \mathbb{R} $ if}
    \begin{enumerate}
        \item 
            \[
                f(x) =
            \begin{cases}
                x sin(\frac{1}{x}) & if x \ne 0\\
                0 & if x = 0
            \end{cases}
            .\] 
        \item 
            We will use our knowledge of the fact that if $f,g : E \rightarrow \mathbb{R} $ are continuous then $fg : E \rightarrow \mathbb{R} $ is continuous.
            Additionally if $f: E \rightarrow E'$ and $g: E' \rightarrow E''$ both continuous then $g \circ f : E \rightarrow E''$ continuous.\\
            $f: \mathbb{R} \rightarrow \mathbb{R} $ with $f(x) = x$ is continuous because for any $\epsilon > 0$ pick $\delta = \epsilon $ and then
            $|x_1-x_2| < \delta $ implies $|f(x_1) - f(x_2)| = |x_1-x_2| < \delta = \epsilon $ satisfying the continuity for all $x_1 \in \mathbb{R} $ and so $f(x) = x$ is continuous.
            \\
            For this question we are allowed to assume that $sin(x)$ is continuous.\\

            For $f(x) = \frac{1}{x}$ at all points $x_0 \ne 0$ and all $\epsilon > 0$. Take $\delta < \text{min}(\frac{|x_0|}{2}, \frac{\epsilon|x_0|^2}{2})$ then we have
            for all $x$ satisfying
            \begin{align*}
                |x-x_0| &< \delta \\
                |x_0x - x_0^2| &< |x_0|\delta \\
                |x_0x| &> |x_0|^2-|x_0|\delta \\
                       &> \frac{|x_0|^2}{2}\\
                \frac{1}{|x_0x|} &< \frac{2}{|x_0|^2}
            \end{align*}
            and so\\
            \[
            |\frac{1}{x}-\frac{1}{x_0}| = \frac{|x_0-x|}{|x_0x|} < \frac{\delta}{|x_0x|} < \frac{2\delta}{|x_0|^2} < \epsilon 
            .\] 
            note that the sign of $x,x_0$ is the same as well. and so for any $x_0 \ne 0$ $f(x)=\frac{1}{x}$ is continuous at $x_0$ and so $f(x)$ is continuous at these points.
            Similarly the composition $sin(\frac{1}{x})$ is continuous at all $x \ne 0$ and $xsin(\frac{1}{x})$ is continuous at all $x \ne 0$\\
        \item 
            \[
            f(x) =
            \begin{cases}
                0 & x \in \mathbb{R} \setminus \mathbb{Q} \\
                \frac{1}{q} & x \in \mathbb{Q} 
            \end{cases}
            .\] 
            with $q$ being the denominator of the simplified fraction representing x.
            Every $x \in \mathbb{Q} $ is not a limit point because $f(x) \ne 0$ and there exists a sequence of irrational numbers
            that approaches any rational number so we can choose $(x_n)_n$ such that $x_i \in \mathbb{R} \setminus \mathbb{Q} $ and $\lim_{n\to\infty}x_n = x$.
            But each $f(x_n) = 0$ so $\lim_{n\to\infty}f(x_n) = 0 \ne f(x)$. This implies that $f(x)$ is not convergent at all rational numbers.\\
            I propose that $f(x)$ is convergent at all irrational numbers.\\
            Take any $\epsilon > 0$ and $x_0 \in \mathbb{R} \setminus \mathbb{Q} $. Clearly $f(x_0) = 0$ and so I must show that for all $x$ s.t. $|x-x_0| < \delta $ for some $\delta $ 
            $f(x) < \epsilon $ because $f(x)$ is strictly positive. First I set $\delta < \text{min}(x_0-\lfloor x_0 \rfloor, \lceil x_0 \rceil - x_0)$ both values are strictly greater than $0$ because
            $x_0$ is not an integer. I then select $N \in  \mathbb{N} $ s.t. $\frac{1}{N} < \epsilon $.\\
            I then note that for all $d \in \mathbb{N} $ and all intervals $(a,a+1)$ with $a \in \mathbb{Z} $ that the number of Rational numbers
            of the simplified (reduced) form $\frac{r}{d}, r \in \mathbb{Z} $ with $\frac{r}{d} \in (a,a+1)$ is less than or equal to $d$. For if there were more than $d$ then for some
            $r_0 \ne 0$ there would be both $\frac{r_0}{d} and \frac{ r_0+d}{d}$ in the set and $\frac{r_0+d}{d} = \frac{r_0}{d} + 1$ which cannot be the case because the interval $(a,a+1)$ does not contain any other elements $b,b+1$ with $b \ne a$.\\
            And so the set $S_d := \{\frac{r}{d}, r \in \mathbb{Z}, \frac{r}{d} \in (\lfloor x_0 \rfloor, \lceil x_0 \rceil)\}$ is finite. And since $N$ from before is finite so to is the set
            $S := \bigcup_{i=1}^N S_i$ is finite.\\
            Since each $s \in S$ is in $ \mathbb{Q}  $ it follows that $s \ne x_0$ and $d(s,x_0) > 0$ and so $ g = \text{min}\{d(s,x_0), s\in S\}$ exists and is greater than zero.
            Choosing $\delta < \text{min}(g,x_0-\lfloor x_0 \rfloor, \lceil x_0 \rceil - x_0)$. We have that all points $x$ s.t. $|x-x_0| < \delta $ are not in $S$ and so they are not rational with denominator less than $N$ and so
            they are either irrational or rational with denominator greater than N and so $f(x) < \epsilon $. Thus $f(x)$ is continuous at all irrational numbers.
    \end{enumerate}
    \section{Exercise 3.7}
    \emph{Let $E, E'$ be metric spaces, $f : E \rightarrow E'$ a continuous function.\\
    (i) show that if $S$ is a closed subset of $E'$ then $f^{-1}(S)$ is a closed subset of $E$ \\
(ii) Show that if $E' = \mathbb{R} $ then $\{p \in E: f(p) \ge 0\}, \{p\in E: f(p) = 0\}$ are closed}
    \begin{enumerate}
        \item $i$ \\
            We know that if $U$ is open in $E'$ then $f^{-1}(U)$ is open in $E$ so  $S' := f^{-1}(S^{c})$ is open. $S'^c$ is closed.
            Now show that $S'^{c} = f^{-1}(S)$ for take any element $x \in S'^{c}$ then $x \not\in f^{-1}(S^{c})$ and $f(x) \not\in S^{c}$ so $f(x) \in S$ and so  $x \in f^{-1}(S)$.
        Similarly take $y \in f^{-1}(S)$ so $f(y) \in S$ and $f(y) \not\in S^{c}$ so $y \not\in f^{-1}(S^{c}) \rightarrow y \in (f^{-1}(S^{c}))^{c} = S'^{c}$ thus $f^{-1}(S)$ is closed.
    \item $ii$ \\
        The sets $\{x : x \ge 0 \}$ and $\{0\}$ are closed in $ \mathbb{R} $ so by the previous argument their preimages are closed.
    \end{enumerate}

    \section{Exercise 3.8}
    \emph{Let U, V be non empty open intervals in $ \mathbb{R} $ and let f : U → V be a
    function onto V which is strictly increasing.\\
    (i) Prove that $f$ is continuous and $f^{-1}: V \rightarrow U$ is continuous\\
    (ii) prove that $f(U')$ is an open interval for every open interval $U' \subset U$.}\\
    Since $f$ is strictly increasing and onto it must also be one-one for if $f(x) = y$ then  $f(x+\epsilon ) > y$ for all $\epsilon > 0$.
    For any $y \in V$ and $\epsilon > 0$ $f^{-1}(\{y-\epsilon, y, y+\epsilon \})$= $\{x_0-a, x_0, x_0+b\}, a,b \in \mathbb{R}^{+}$ by the fact that $f$ is strictly increasing and onto.
    thus choose $\delta < \min({a,b})$ and so if $x$ satisfies  $|x-x_0| < \delta  $ then by the ordering of sets it must be in the interval $(x_0-a,x_0+a)$ and since the endpoints corresopnd to the endpoints of the image interval bounded by $\epsilon $ and since $f$ is strictly increasing
    all of these points must fall into the $\epsilon $ interval and $|f(x) - f(x_0)| <  \epsilon $.\\
    The inverse function is also continuous because the inverse function is also strictly increasing and onto and one-one from one interval to another interval and so it satisfies the exact same properties.
    \\
    $ii$ Because $f$ is continuous and $f^{-1}$ is continuous on $U$ for any open set $O \subset U$ $f(O)$ is open in $V$ for $O$ for $f(O)$ is the preimage of an open set under the continuous function $f^{-1}$.
    Assume for some open interval $U' \subset U$ that $f(U')$  is not an interval. Since it is both open and not an interval there must exist some point $c \in V \; c \not\in f(U')$ such that there exists $a,b \in f(U')$ with $a < c < b$. but $a' = f^{-1}(a) < c' = f^{-1}(c) < b' = f^{-1}(b)$ which implies that $c' \in (a',b') \subset U'$ since $U'$ is an interval a contradiction.

    \section{Exercise 3.9}
    \emph{Let $E$ be a metric space, $S$ a nonempty subset of $E$, and let $f : E \rightarrow \mathbb{R} $
    be the function which takes the value 1 at each point of $S$ and the value 0 at each point of
    $S^c$. Prove that the set of points of $E$ at which $f$ is not continuous is precisely the boundary
    of $S$.}
    By definition of the boundary of $S$ every open ball at a point on the boundary contains points both inside the set and outside. So for any $x_0 \in bd(S)$ and any $\delta > 0$ there exists $s_1,s_2 \in D_\delta(x_0)$ with $s_1 \in S$ and $s_2 \in S^{c}$ and so if $\epsilon < 1$ $d(s_1,s_2) > \epsilon $ for all $\delta $.
    For all of the other points that are not on the boundary they are strictly in the interior of either set and following similar logic there exists some $\delta > 0$ with $D_\delta(x_0)$ contained entirely in the set for $x_0$ in the interior of either $S$ or $S^{c}$.
    And so for any $\epsilon > 0$ and any $p_1,p_2 \in D_\delta(x_0)$ $f(p_1) = f(p_2) \rightarrow d(f(p_1),f(p_2)) = 0 < \epsilon $ and so the function is continuous at those points.

    \section{Exercise 3.10}
    \emph{(i) Prove that if $S$ is a nonempty compact subset of a metric space $E$ and $p_0 \in E$ then $\min\{d(p_0,p) : p \in S\} $ exists.\\
        (ii) Prove that if $S$ is a nonempty closed subset of $ \mathbb{R}^{n}$ and $p_0 \in \mathbb{R}^{n}$ then $\min\{d(p_0,p) : p \in S\}$ exists.
    }
    \begin{enumerate}
        \item $(i)$ \\
            Since $S$ is compact it contains all of its limit points. Consider x =   $\sup\{d(p_0,p) : p \in S\}$ This exists because it is bounded below by $0$ and so there exists some sequence $(x_n)$ s.t
            $d(x_n,p_0) \rightarrow x$. And since $k$ is compact there is a subsequence $(x_{n_k})_k$ that converges to a point $x' \in S$. and since subsequences have the same converges as sequences,
            $d(x_{n_k}, p_0) \rightarrow x$ and so $d(x',p_0) = x$ Thus $x'$ satisfies the $\min\{d(p_0,p) : p \in S\}$

        \item $(ii)$
            By the fact that all nonempty closed and bounded subsets of  $ \mathbb{R}^{n}$ are compact the result follows from the previous answer in the case where $S$ is bounded.
            If $S$ is unbounded take $S \cap B_\epsilon(p_0)$ the intersection of $S$ with a closed ball of some radius  $\epsilon $ such that the intersection is nonzero. Then this new set is closed
            and the closest element in this set is the same as the closest element in the starting set because all elements of the starting set that are not in the new constructed set are at least  $\epsilon $ away from $p_0$ whereas
            all of the elements in this new set are less than that distance and thus are a smaller distance.
            
    \end{enumerate}

    \section{Exercise 3.11}
    \emph{Let $E, E'$ be metric spaces, $f: E \rightarrow E'$ a continuous function. Prove that if $E$ 
    is compact and $f$ is one-one onto then $f^{-1}: E' \rightarrow E$ is continuous.}.\\
    Consider any sequence in $E'$, $(x_n)_n$ converging to $x_0$. Because $f$ is one-one onto each $x_i$ corresponds to some $f(p_i)$ with  $p_i \in E$.
    We can rewrite the sequence then as $(f(p_n))_n$. Consider the corresponding preimage sequence $(f^{-1}(f(p_n)))_n$ in $E$.
    By compactness of $E$ there exists some subsequence $(f^{-1}(f(p_{n_k})))_k$ that converges to lets say $p_0$ in $E$.
    By one-one onto each $f^{-1}(f(p_i)) = p_i$ so the sequence $(p_{n_k})_k \rightarrow p_0$.
    And by continuity $(f(p_{n_k})) \rightarrow f(p_0)$.
    Because all subsequences of convergent sequences converge to the same limit $(f(p_{n_k}))_k \rightarrow x_0$. This implies that $x_0 = f(p_0)$ and $f^{-1}(x_0) = p_0$. Thus for every sequence in $E'$ converging to some $x$. The map of the 
    sequence by $f^{-1}$ also converges to $f^{-1}(x)$ which implies that $f^{-1}$ is continuous.

    \section{Exercise 3.13}
    \emph{Let $S$ be a subset of the metric space $E$ with the property that each point
    of $S^{c}$ is a cluster point of $S$ (one then calls $S$ dense in $E$). Let $E'$ be a complete metric
    space and $f : S \rightarrow E'$ a uniformly continuous function. Prove that $f$ can be extended to a
    continuous function from $E$ to $E'$ in one and only one way, and that this extended function
    is also uniformly continuous.}\\
    For $f$ to be continuous it must be sequentially continuous. This implies that every sequence converging to some point in $E$ say $(x_n)_n \rightarrow x$ must converge to the same point  $f(x)$ in $E'$.
    To extend $f$ to $E$ it suffices to define $f$ on all points not in $S$.
    Take any point $x_0 \in S^{c}$. Since $x_0$ is a cluster point we can construct a sequence $(x_n)_n$ with each $x_n \in S$ and   $d(x_i,x_0) < \frac{1}{i}$ for all $i \in \mathbb{N} $.
    The sequence obviously converges to $x_0$. consider the sequence $(f(x_n))_n $. By continuity and the fact that every convergent sequence is a cauchy sequence,
    for any $\epsilon $ we can pick $\delta $ such that for some $N \in \mathbb{N} $, $d(x_n,x_m) < \delta  $ for all $n,m > N$ and $d'(f(x_n), f(x_m)) < \epsilon $ by continuity. Thus
    the sequence $(f(x_n))_n$ is Cauchy and since $E'$ is complete this sequence converges to a point which we set to be  $f(x_0)$. It is important to note that this element does not depend on the sequence of $x_n$ used because $f$ is uniformly continuous.
    By this formulation every sequence that converges to an element
    of $S^{c}$ has the same limit and is continuous. If for any $x' \in S^{c}$ a different convergence point existed then $f(x') \ne \lim_{n\to\infty}(f(x_n))_n$ as described as above and so the function would not be continuous.

    \section{Exercise 3.15}
    \emph{A metric space $E$ is said to be arcwise connected if, given any $p,q \in E$, there exists
        a continuous function $f : [0,1] \rightarrow E$ such that $f(0) = p, f(1) = q$. Show that\\
        (i) an arcwise connected metric space is connected\\
        (iii) any connected open subset of $R^{n}$ is arcwise connected.
    }
    \begin{enumerate}
        \item $(i)$ \\
            Assume that $E$ is arcwise connected and for the sake of contradiction it is not connected.\\
            Then there exists two sets $A,B$ s.t. $A \cap B = \emptyset$, $A \ne \emptyset, B \ne \emptyset$ and $A \cup B = E$.
            Since both $A,B$ are nonempty there exists some $a,b \; a \in A, b \in B$. And by 
            arcwise connectedness there exists a function $f_{a,b} : [0,1] \rightarrow E$ s.t. $f(0) = a$ and $f(1) = b$.
            Since $f$ is continuous and $[0,1]$ is a connected set in $ \mathbb{R} $. $U =$$f([a,b])$ is a connected set in $E$ that includes
            $a$ and $b$. But  $(A \cap U) \cup ( B \cap U) = (A \cup B ) \cap U = U$ with $(A \cap U)$ and $(B \cap U)$ both closed which is 
            a contradiction because $U$ is connected and therefore not a union of two disjoint closed sets because $(A \cap U ) \cap (B \cap U) = (A \cap B) \cap U = \emptyset \cap U = \emptyset$. Thus  $E$ must be connected.
        \item $(iii)$ \\
            First note that in $ \mathbb{R}^{n}$. All open balls are arcwise connected for take an open ball $D_r(x)$ and any two points $x_1,x_2 \in D_r(x)$ then the function $f(t) : [0,1] \rightarrow R^{n} = x_1t + x_2(1-t)$ is an arc connecting the two points that is entirely
            contained in the open ball. Assume that $S \subset \mathbb{R}^{n}$ is connected and open but not arcwise connected for the sake of contradiction.
            then there exists two poitns $x_0,x_1 \in S$ such that there is no continuous function that maps between them. Consider the set $U$ of all points that are arcwise connected to $x_0$.
            $U$ is clearly open in $S$ because each point in $U$ there exists an open ball of radius $\epsilon $ such that the ball centered at the point with radius $\epsilon $ is contained in $U$ by the previous argument and contained in $S$ because $S$ is open.
            Similarly $U^{c}$ is open for each $p \in U^{c}, S$ there exists some $\epsilon > 0$ s.t. $D_\epsilon(p) \subset S$ and by the argument about open balls every point in that ball is in $U^{c}$ for otherwise the point would be in $U$ and then there would be a path between that point and our starting point.
            Thus we have constructed  $U, U^{c}$ with $U \cap U^{c} = \emptyset, U \cup U^{c} = S$ and $U, U^{c}$ open violating the fact that $S$ is connected. Therefore we have a contradiction and all connected open subsets of $ \mathbb{R}^{n}$ are arcwise connected.
    \end{enumerate}





\end{document}
